\hyphenation{na-cio-nal ins-ti-tu-cio-nais ori-en-te ita-lia-na vio-len-tos Eduar-do con-cen-tra-cio-ná-rio prio-ri-ta-ria-men-te fran-quea-do rea-li-zei rea-li-za-da atual-men-te gra-dual-men-te}


\chapter*{}

\section{Crise real e \emph{pequena
  história}}\label{crise-real-e-pequena-histuxf3ria}


O que se segue é \emph{uma pequena história política}. Ela tenta
identificar os elementos principais do que se tornou o esfacelamento do
mundo petista de Dilma Rousseff, no início do quarto mandato
presidencial consecutivo do Partido dos Trabalhadores.

São \emph{fatores} de política -- de \emph{política tradicional --} que
serão lembrados aqui. Esta história recente, delimitada por \emph{um
ponto de vista do presente} e evocada em imagens condensadas do
processo, no momento mesmo do seu gume mais afiado, também busca
completar algo do desenho anterior que realizei a respeito do político
Luis Inácio Lula da Silva.\footnote{\emph{Lulismo, carisma pop e cultura
anticrítica,} São Paulo: Hedra, 2011.} Ele
poderá ser observado, agora, pelo prisma dos resultados trágicos de suas
próprias decisões, quando tomadas de modo que poderíamos chamar
\emph{narcísico}, e considerando-se o sentido das coisas que cercaram o
governo Dilma Rousseff também como parte da ação política do próprio
ex-Presidente. Este processo histórico, mesmo que \textit{a posteriori}, ainda
realiza uma última faceta de seu semblante.

Todavia, é preciso dizer que não acredito substantivamente em tudo o que
vou elencar neste trabalho. Não quero dizer que estas \emph{entidades}
políticas e sociais não existam. Elas existem, andam pelas ruas, são
frutos de imensas violências e também produzem a própria nova ordem de
violências. Mas penso que o centro de um verdadeiro trabalho político
de hoje deveria ser deslocado para \emph{outra pista}, que embora
totalmente real, ela própria não encontra espaço prático para existir.
E, deste modo, acabamos uma vez mais no mundo das \emph{pequenas
categorias}, que dão substrato para a \emph{pequena história}, leitura
do real que tenta repor uma ordem imaginária programática, forma
interessada por princípio, bem orientada para o poder e seus termos.
Isso enquanto o próprio poder – se olhado de sua dinâmica mundial, já livre de
qualquer amarra e em velocidade entrópica – destrói constantemente as próprias
medidas locais que ainda necessitamos para situá-lo.

O principal da crise mundial que de fato vivemos, e que por aqui acabou
por se precipitar como este nosso teatro nacional de impotências -- nada
inteligente, o que nos dá boa medida de onde estamos, com incríveis
evocações do primitivo autoritarismo brasileiro, em uma espécie de
falência cidadã do uso da própria democracia, que, ao que tudo indica,
além de mero sinal de desespero é ainda menos do que farsa, teatro
becktiano de eterno retorno do nada -- é, sobretudo, acredito, uma real
impossibilidade atual do Capital mundial em se reproduzir sem, de
imediato, destruir tantas realidades, países, sociedades, espécies, o
planeta, psiquismos, corpos, de modo que, simplesmente, ele vai se
tornando cada vez mais um impasse universal à simples manutenção da
própria regra do seu jogo. A grosseria autodestrutiva humana vista nas
ruas do país, que clama por mais capitalismo à brasileira, também é um
momento claro desta universalidade do poder de destruição randômico de
nossa época. Mas, por outro lado, não devemos nos enganar, \emph{alguém
deve pagar a conta} de um processo de acumulação que não cessa.

O que menos vale, no nosso curioso e menor caso
\emph{periférico-central}, é o modo de mascarada próprio ao tempo, o
nosso \emph{transe} específico, com que respondemos àquilo que importa:
a realidade de uma \emph{terra em transe} em sua própria verdade, aquela
que, mais uma vez, foi pensada entre nós por Paulo Eduardo Arantes:

\begin{quote}
``A entropia avassaladora de agora não afeta apenas os últimos doze anos
e meio de hegemonia lulista como se costuma resumir o polo prevalecente
nesse período de Fla\,x\,Flu eleitoral ininterrupto, mas o longo prazo
iniciado por uma Transição que está morrendo agora na praia. Os
pretensos herdeiros desse espólio simplesmente não sabem o que os espera
ao apressarem seu fim institucional. Estarão abreviando sua própria
sobrevida, pois a fuga para frente que ainda insistimos em chamar de
crise é antes de tudo um processo ao qual nenhum lance dramático porá
fim, nem suicídio, quanto mais intrigas regimentais de políticos e
negocistas de quinta. Vencidos pelo cansaço, então, também é isso:
trinta e cinco anos ralando, e de permeio um descomunal desperdício
{[}\emph{do \versal{PT}}{]}, o próprio emblema da tragédia segundo os Antigos.

A crise é assim, essa convergência desastrosa de uma inédita exaustão de
todo tipo de recursos, dos mais elementares aos mais elevados, da
polinização à imaginação política. Até a potência de Junho parece que se
esgotou. Pois é tal a entropia do capitalismo, desorganizado desde o Big
Bang de meados dos anos 1970 em seu núcleo orgânico, que desorganiza até
mesmo as forças antissistêmicas. Só para efeito de comparação, veja-se o
caso do outrora maior partido de esquerda do Ocidente. O \versal{PT} não está
agonizando por força de rejeição imunológica, por maior que seja o
efeito do choque externo das ondas sucessivas de anticorpos enraivecidos
até o ódio mortal, mas por motivo de uma combustão interna que o
consumiu, por assim dizer, do berço ao túmulo. Nenhum ato de violência
de classe o desviou de sua vocação original, pura e simplesmente
dissipou-se a energia que o mantinha em funcionamento. Bem como a das
grandes centrais sindicais e movimentos sociais históricos que
gravitavam em sua órbita. Foram todos vencidos pelo cansaço, como sabe
todo batalhador de movimento social, quase sempre à beira de
um~\emph{burnout}.''\footnote{``A fórmula mágica da paz social se
  esgotou'', Correio da Cidadania, 17/07/2015.\textsuperscript{\\}}
\end{quote}

Apenas para o conhecimento de um teatro de máscaras contemporâneo, que
vai dos indivíduos no poder, disputando qualquer coisa, às massas na
rua, clamando por qualquer coisa -- passe livre, fora \versal{\versal{PT}}, contra a
corrupção, contra o comunismo, por ditadura, aqui não é a Venezuela…
-- me dediquei um pouco às figuras de processo histórico que se seguem.
Uma vez que, será ainda deste repertório mais ou menos qualquer,
aleatório, do passado falido, mas ainda ativo, de nossa humanidade que
se buscará, outra vez, repor os pactos, os sentidos que garantam alguma
\emph{rósea luz} -- ao determinar quais serão os sacrificados da vez, e
quem os escolherá -- sobre a verdade, ainda intolerável, de nosso
verdadeiro \emph{tempo sombrio}.

  \section{Política e mágica}\label{poluxedtica-e-muxe1gica}

Há muito intuímos a existência de sortilégio, encantamento ou pensamento
mágico na política. Embora este tipo de matéria humana seja considerada
por tantos, e por certo tipo de cientista da política, como beirando o
irreal, porque irracional, ou como sendo da ordem das coisas
pré-políticas, por exigir contato com as forças desejantes e imaginárias
humanas, ela é de fato bastante viva e de grande eficácia na vida
concreta das coisas do poder.

Sabemos, por exemplo, que a convocação e a dominação \emph{carismática}
do político é baseada na possibilidade da captura de nosso desejo por um
corpo, uma personalidade, um estilo, um ritmo e uma voz, de forma que,
para além dos elementos tradicionais que podem ser codificados --
econômicos e sociais -- estes dados estéticos, porém inconscientes,
também fundam uma esperança, e geram uma energia política, sobre e a
partir de aspectos imaginados a respeito das qualidades do governante. O
que sonhamos do líder, e, principalmente, o seu modo único de nos fazer
sonhar -- com os complexos pactos de \emph{comunicação} que a ele se
agregam -- em conjunto com o que ele pode ou não entregar, faz parte do
valor de dominação que ele exerce sobre nós. E o caso exemplar extremo
de Lula não nos deixa enganar sobre este ponto.

Embora esta dimensão das coisas seja comum, ela é também um dos maiores
segredos da política. E sabemos, desde sempre, que algo do poder de Lula
passou pelo momento político mágico da sua convocação do amor amplo de
grande parte do público brasileiro por ele. Bem como, o que importa
agora, o ódio brutal que se expressa hoje nas ruas do país contra o \versal{PT} é
também a tentativa astuciosa e igualmente mágica, baseada em grandes
emoções e na redução calculada da linguagem, de anular e esvaziar os
motivos encantatórios daquele monumental amor dos brasileiros pelo
ex-Presidente.

E, sabe-se, e isto \emph{é simplesmente assim}, que Dilma Rousseff
jamais teve nenhum tipo de poder parecido, nenhum poder de atração e de
sedução ligado ao seu corpo e ao seu estilo duro de fazer política. Bem
ao contrário, ela muitas e muitas vezes afastou e criou dissenso, até
mesmo entre os próprios homens de seu governo, quanto mais frente aos
seus inimigos. Muitas outras vezes ela apareceu no espaço público com
pequenos mas bem nítidos sinais de arrogância e impaciência e ainda em
muitas outras oportunidades -- o que para mim sempre foi o mais difícil
em minha própria relação com a Presidente -- teve muitas dificuldades em
tornar apenas claras as próprias declarações. Os poderes do carisma
político nunca estiveram entre as maiores qualidades da Presidente
petista.

Curiosamente, sobre este aspecto, \emph{as funções de personalidade}
próprias do político Lula, que produziam os efeitos conciliatórios e de
busca de consenso, estavam muito mais próximas das habilidades do seu
célebre inimigo íntimo, Fernando Henrique Cardoso -- embora os dois
sejam homens de estilos muito diferentes -- do que da própria pupila,
política tecnocrata de esquerda, escolhida para sucedê-lo. Olhando deste
ponto de vista, podemos dizer que Dilma Rousseff não fazia \emph{sombra
alguma} a Lula. Por outro lado, pela orientação da política e pela
``maneira de lutar'', Dilma se aproximou mais, ao menos em seu primeiro
mandato, das formas próprias das garantias do poder à brasileira
\emph{que deseja ser forte}, com traços de autoridade autoritária, que
podemos recordar presentes no Brasil em um… Ernesto Geisel, por
exemplo.

Tudo indica que, em sua muito singular carreira política, a única pessoa
que Dilma Rousseff parece um dia ter de fato seduzido foi o próprio
Lula. Ele mesmo, o mago que transformava coisas inertes, \emph{postes},
em verdadeiros seres vivos -- \emph{golens} diziam os antigos místicos
judeus -- que se tornariam seres políticos como uma espécie de
continuidade de seu próprio sonho e projeto. O novo ``rei taumaturgo''
que, a esta altura dos acontecimentos, talvez se arrependa da certeza de
sua percepção sobre a própria sucessora, realizada no auge mais intenso
do próprio poder, do próprio auto-encantamento de Lula.

E além desta escolha, Lula também prometeu a si mesmo, e a todos os
brasileiros, que o país jamais seria atingido pela onda gigante da real
crise mundial do capitalismo, construída em Wall Street, e que, de 2008
até hoje, desempregou meio mundo, literalmente. No Brasil, aquele
tsunami das finanças mundial se tornaria dócil, como as ondas que chegam
à beira das praias de um paraíso -- de consumo -- tropical. Vinda de
Lula, com as conexões oníricas na época existentes com seu corpo, seu
desejo e sua política de economia interna alavancada, o seu
\emph{keyniasinismo de consumo}, tal profecia se tornava automaticamente
meio auto realizada. Pois, como hoje sabemos, ela simplesmente não se
realizou. Certamente deveria haver limite histórico para tanto desejo e
gosto pela mágica, autoconfiança na própria prestidigitação e simultâneo
déficit de lucidez.

Na tábula rasa da política lulista, e com o sucesso econômico de seu
mundo de carisma pop -- também baseado na reprodução do velho Brasil na
política -- Lula chegou ao fim de seu governo podendo escolher qualquer
um como seu sucessor. E esta era uma posição heterodoxa na política de
forças presentes em uma democracia complexa. Com a ruína do seu grupo de
políticos próximos, atingidos em pleno voo no processo penal do mensalão
petista, Lula restou com o seu arbítrio final, a fantasia política
extrema de \emph{determinar sozinho} quem seria sua sucessora no próprio
poder de mando nacional… Algo se dava \emph{exatamente como a escolha
dos sucessores dos antigos generais presidentes,} durante a ditadura de
1964--1984, como um gesto privativo do detentor extremo do poder… --
cujo resultado, \emph{exterior à vida social}, implicava na época de
fato um quadro de ditadura, de extrema concentração de poder, para
chegar mais ou menos a funcionar -- …

E, de algum modo, Lula escolheu qualquer um, \emph{uma companheira que
jamais passara por sequer uma eleição} \emph{majoritária}. Seu desejo de
que não existisse sombra política no \versal{PT} a ele próprio, desejo de mais e
máximo poder em um partido que praticamente se dissolvera nele, e por
ele, levou à sua radical escolha de alguém com reais e grandes
dificuldades políticas. Desde a origem, e sempre, a tecnocrata de
esquerda escolhida, advinda do \versal{PDT} de Leonel Brizola, representou
instabilidade política muito forte no próprio \versal{PT} de Lula. E afora o
pacto, também gerencial, pragmático -- que acabou explodido pela justiça
do juiz Sérgio Moro -- com as grandes empreiteiras envolvidas no projeto
geral da tentativa de desenvolvimentismo, ainda não se consegue pensar a
que forças sociais Dilma Rousseff estava ligada ou representava afinal
-- e, precisamente sobre este aspecto, a sua condição se parecia um
tanto com a de Fernando Collor de Mello, um presidente
\emph{impichado}…

Assim, no auge do poder, começava-se a plantar a semente da derrocada,
como um gesto inteiramente próprio e pessoal do líder auto-encantado. 
Todavia, o que foi se revelando uma fuga para frente de Lula poderia
encontrar um limite muito maior do que toda a autoconcepção e o
narcisismo, que a moviam, podiam conceber.

  \section{Dilma Rousseff:
  \emph{anti-príncipe}}\label{dilma-rousseff-anti-pruxedncipe}

E esta evocação do passado autoritário brasileiro, que não passa, de
grande concentração de poder em figuras convencidas que política deve
ser feita deste modo -- agora realizada na líder tecnocrática de
esquerda, braço direito da organização da administração lulista, ex-jovem 
combatente contra a ditadura, economista de partido, que
transformou política em gerência, que parece fantasiar uma certa batalha
imaginária, quadro técnico em um mundo sem quadros técnicos -- é figura
híbrida brasileira de interesse, por seu melancólico mas ativo
anacronismo, ao final de tudo, por fim, claramente impotente.

Olhando nossa pequena tragédia pelo ponto de vista das funções de
personalidade do político, interessantemente, José Serra, o candidato
alternativo eterno do \versal{PSDB} ao projeto petista, era também um personagem
bastante semelhante a este amálgama de vagas noções à esquerda e mais
forte consciência e desejo de controle da economia, visando a desenhar o
mundo algo ao seu próprio modo. Dilma e Serra, o seu adversário primeiro
na eleição de 2010, eram igualmente ativos, interventores, igualmente
herdeiros de uma fantasia de técnica e de razão forte para a gestão do
Estado no Brasil. Serra representava, de fato, uma espécie de
\emph{neo-desenvolvimentismo} tucano -- em uma última imagem muito
esmaecida de um Celso Furtado, um moralista técnico, com compromisso com
o país, em meio à nova irresponsabilidade financista mais geral da
própria classe e partido -- e não seria nada impossível que um governo
hipotético seu terminasse por se ancorar precisamente no mesmo lastro do
Capital local que sustentou o governo petista de Dilma Rousseff: as
grandes empreiteiras nacionais, e seu modo perverso próprio de jogar com
o Estado e a política. Elite tecnocrática, agora dita de esquerda, era
de fato o mais amplo e antigo positivismo brasileiro, transfigurado em
agenciamento técnico com o desenvolvimento industrial da segunda metade
do século \versal{XX}, que ainda falava nestes personagens, aparentemente
formados no processo limite da política nacional, nos anos de 1960 e
1970, e sua dialética para o nada.

Eram tecnocratas de esquerda, e Dilma era, até mesmo pelo grande mau
humor e a tendência forte ao dissenso, um certo tipo de Serra de saia.
Ou, mais aproximado pelo partido, pelo próprio Estado que representa e
por Fernando Henrique Cardoso, ao capitalismo financeiro brasileiro, os
núcleos locais de gestão de fundos globais, José Serra era também uma
espécie particular de Dilma de calça. Não é o pitoresco da analogia que
importa, mas a própria duplicação. Há um padrão nesta impressão não
superada, de um certo passado, dado no modo ainda assertivo destes
homens serem. O Brasil teria que, necessariamente, reviver as ações
econômicas fortes de governo ativo, mas conservador, destes homens e
mulheres que, queiramos ou não, tiveram as formações marcadas e suas
ações definidas, em grande parte da própria vida, pelo projeto de
construção de Império sobre os despojos da vida social da ditadura
militar de 1964/1984. Talvez ambos, como tantos outros, fossem marcados
por alguma identificação com o agressor fundamental de suas próprias
histórias. O espírito, tão tipicamente brasileiro, de general, de
\emph{quem não gostar, eu prendo e arrebento}, podia ser ainda evocado,
mesmo que matizado, em algum momento da personalidade de um ou da outra.

No plano do cotidiano, do discurso político da vida, nada poupava os
dois personagens de serem vistos como irascíveis e, malgrado ou não o
próprio processo histórico mais amplo das forças políticas que mais ou
menos representavam, comumente autoritários. E, assim, sobre o primeiro
período Dilma de governo, de 2011 a 2014, não era nada difícil
esbarrarmos com ressentimentos e reclamações a respeito do gênio e dos
modos gerais do comportamento político da Presidente. Ficamos sabendo,
por esta dinâmica social comunicativa humana que escapa aos controles de
governos, e vem de todos os lados, que qualquer coisa podia mesmo
acontecer na sua presença, menos as manhas e artes da cordialidade
estratégica lulista, ou da diplomacia elevada e satisfeita de
\emph{chics} entre si, de tipo Nabuco, de \versal{FHC}.

Por isto, de tempos em tempos, quando enfiada em significativas crises
de negociação, sempre aumentadas por ela própria, Dilma Rousseff
irremediavelmente precisou pedir lições de etiqueta, de saídas
propagandísticas de imagem e de manejo político, a Lula. Lições que,
tudo indica, ela nunca levou a sério. Além disto, não sendo Lula, e Lula
não estando no lugar do jogo político dela, as intuições do
ex-Presidente de nada lhe serviram. Isto não impediu que, ao fim de seu
governo, ela própria fosse vista como má agradecida e recalcitrante pelo
próprio Lula, e sua família.\footnote{Ver ``A afilhada rebelde'', por
  Daniela Pinheiro, outubro de 2014, revista
  Piauí.} A entropia presente na forma do controle
era grande.

Eu também tive a oportunidade, comum, do andamento normal da vida, de
ouvir reclamações amargas a respeito da vida política e pública com a
Presidente. Essas vozes diziam mais ou menos que ela não parecia manter
contato com o que importava, nem em projetos e nem em cuidados com o
outro. E, se ouvi isto, é certo que muitos outros também ouviram. Aqui
temos a famosa multiplicação das vozes públicas comuns, que corroem o
poder de governos e de príncipes.

Me recordo particularmente, a respeito do modo dela encaminhar sua
política entre os homens, do desalento de um diplomata, ligado a
significativas políticas de Estado lulistas, e, também, da ironia de um
lobista, que representava interesses fortes o suficiente para frequentar
o palácio da Alvorada. Ambos os homens, em tudo diferentes, se referiam
à Presidente, de modo cifrado ou aberto, dependendo do próprio
\emph{habitus} social, com o mesmo acento negativo, e o mesmo horizonte
de desesperança, de real caso perdido…

A inabilidade radical com o outro, \emph{o suposto inimigo} no jogo do
poder e nos interesses da economia, de uma neófita em toda possibilidade
de negociação política ou pública, só era superada pela incrível
inabilidade na lida com os próprios parceiros de poder, com as
constantes escaramuças, divididas e rupturas com os próprios homens, a
própria equipe de governo. Alguns pequenos exemplos, do modo que, reza a
lenda, se tornou sistemático de Dilma Rousseff gerir -- afinal, segundo
a ideologia que a sustentou, ela, mais do que uma política e menos do
que uma governante, era uma \emph{gerentona…} uma palavra muito
feia… \emph{--} que parece ter acontecido em todos os níveis da sua
ação, da burocracia oficial aos embates com os grande interesses:



\begin{quote}
``Depois de ter
sido {desautorizado
publicamente pela Presidente Dilma Rousseff a emitir opinião sobre o
caso Vivo/\versal{\versal{TIM}}}, o ministro das Comunicações, Paulo Bernardo, mudou o
discurso. Ele disse nesta segunda-feira que `não devemos ficar falando
(sobre o assunto) porque essa briga não é aqui no Brasil.' (…) Na
semana passada, Bernardo disse que resultaria em concentração `muito
grande' no mercado de telefonia brasileiro o aumento da participação da
dona da Vivo (Telefônica) no capital da dona da \versal{\versal{TIM}} (Telecom Italia). A
Presidente Dilma reagiu. `Houve uma opinião do ministro Paulo Bernardo
(das Comunicações), que não é a opinião oficial do governo, ainda' --
afirmou ela na quarta-feira passada, em Nova York, após reunião da
Organização das Nações Unidas (\versal{\versal{ONU}}). A Presidente afirmou que caberá ao
Conselho Administrativo de Defesa Econômica (Cade) se manifestar sobre o
acordo. `Ela está totalmente certa. O Cade tem que examinar a
concentração de mercado' -- disse Bernardo nesta segunda-feira.''
\end{quote}

\asterisc



\begin{quote}
``O ex-articulador político do governo, Pepe Vargas, convocou uma
coletiva de imprensa na tarde desta quarta-feira para anunciar que foi
convidado pela Presidente Dilma Rousseff para ser ministro da
Secretaria de Direitos Humanos da Presidência da República e aceitou a
proposta. Em seguida, após ser interrompido por uma ligação, ele recuou
das afirmações. `A presidenta Dilma me convidou para ir pra Secretaria
de Direitos Humanos, coloquei à presidenta que eu poderia ajudar o seu
governo na Câmara dos Deputados. A Presidente insistiu que eu
permanecesse na sua equipe', afirmou. `Sou daqueles que acham que as
pessoas são os seus valores e as suas circunstâncias. Dentro dos meus
valores, acredito que não se deve dizer `não' a um pedido da presidenta
da República. Não tem nenhuma circunstância que me impede de ir para
\versal{SDH}, então pelos meus valores e pela ausência de circunstâncias que
dificultem minha ida para \versal{SDH}, vou acolher então o pedido da
Presidente', havia afirmado logo antes da ligação. Pepe, no entanto,
mudou o tom do discurso depois de receber uma ligação telefônica que
interrompeu por três minutos a coletiva de imprensa. Apesar de ter
afirmado, momentos antes, que Dilma insistiu para que ele aceitasse o
pedido e confirmado que vai `acolher o pedido', Pepe passou a dar
respostas evasivas e insistir apenas que tem condições de colaborar com
o governo Dilma se ela decidir. Ele ressaltou que também não teria
problemas em reassumir o mandato que tem de Deputado federal. `A única
coisa que tenho garantia é o meu mandato de Deputado', disse. De volta à
sala de entrevista, disse que Dilma não confirmou sua nomeação para a
Secretaria. `Não houve um comunicado oficial em relação a isso. O que
digo para a presidenta é que, se ela quiser me aproveitar na sua equipe,
tranquilamente aceito o convite de continuar contribuindo com a sua
equipe', afirmou. Perguntado sobre quais seriam suas prioridades como
chefe da Secretaria de Direitos Humanos, Pepe respondeu: `Não fui
nomeado ministro da secretaria de Direitos Humanos, não há por que eu
fazer esse tipo de comentário'.

Ele foi retirado da Secretaria de
Relações Institucionais da Presidência da República com a decisão de
Dilma de que o vice-Presidente Michel Temer passaria a acumular a função
de articulação política. Questionado sobre a ligação telefônica, o
petista respondeu: `Era um telefonema que eu tinha de atender, que eu
tinha solicitado'. Para Pepe, ao tirá-lo do comando da articulação
política do governo, Dilma fez uma opção de substituir o \versal{PT} pelo \versal{PMDB},
partidos da base do governo que têm protagonizado atritos. `Esse ruído
desorganiza e desestabiliza o conjunto da base', afirmou.''
\end{quote}


\asterisc

\begin{quote}
``Com apenas um dia no cargo, o ministro do Planejamento, Nelson
Barbosa, levou ontem uma bronca da Presidente da República, e, por ordem
de Dilma Rousseff, divulgou nota em que afirma que `a proposta de
valorização do salário mínimo, a partir de 2016, seguirá a regra de
reajuste atualmente vigente'. Pela manhã, depois de ler os jornais na
praia, na base naval de Aratu, na Bahia, onde descansa, a Presidente
ficou bastante irritada com a repercussão das declarações de Barbosa do
dia anterior, sobre a mudança na regra de reajuste do salário mínimo, e
mandou o ministro divulgar uma nota desmentindo as afirmações. (…) A
declaração do ministro foi considerada um desastre político pelo Palácio
do Planalto. Segundo interlocutores de Dilma, houve falta de
`sincronismo político'. A equipe econômica vinha estudando uma nova
fórmula de correção dos rendimentos, com o aval do Planalto, mas o
núcleo político do governo queria que esse assunto viesse à tona somente
no segundo semestre. Com a ordem de Dilma para Barbosa desmentir as
declarações, o governo, na prática, se comprometeu em manter a fórmula
vigente, na contramão do que a equipe econômica pretendia. A nota
assinada por Nelson Barbosa foi divulgada no começo da tarde de ontem:
`O ministro do Planejamento, Orçamento e Gestão, Nelson Barbosa,
esclarece que a proposta de valorização do salário mínimo, a partir de
2016, seguirá a regra de reajuste atualmente vigente. Essa proposta
requer um novo projeto de lei, que deverá ser enviado ao Congresso
Nacional ao longo deste ano'.''\footnote{Rennan Setti, em O Globo,
  30/09/2003; Bernardo Caram, Rafael Moraes e Tânia Monteiro, em O
  Estado de S. Paulo, 08/04/2015 e Gabriela Valente,~Simone
  Iglesias~e~Catarina Alencastro em O Globo, 03/01/2015.}
\end{quote}

Tudo sempre pareceu indicar que, para a Presidente, a sua prerrogativa
de desmentir os homens de seu governo quando entendesse, particularmente
em público, era bem mais importante do que a ideia de harmonia,
coordenação e a virtual lealdade interna do próprio governo. Mistura de
sua necessidade de afirmação da última palavra, de controle de todas as
ações e de demonstrar-se a ativa \emph{matriarca} de seu próprio poder
-- em uma imagem de \emph{self} \emph{político} primitiva --
infantilizando e confundindo os seus agentes de governo e de Estado --
com constantes sinalizações de desorganização e de desencontros
denunciados por ela própria -- tal comportamento gerou o mito da mulher
política autoritária e de acesso improvável, que só fez crescer. Não há
dúvida que, sobre muitos aspectos, ela contribuiu ativamente para o seu
próprio isolamento.

Nos casos trazidos acima, já avançados na história da crise interna ao
governo de Dilma Rousseff, as desavenças públicas com a equipe vão de
questões burocráticas de posicionamento de Estado, envolvendo grandes
interesses empresariais, a quase pitorescas gafes relacionadas ao trato
com a informação e o destino dos próprios homens, até importantíssimas
políticas envolvendo ganhos ou perdas dos trabalhadores no Brasil. E, em
nenhum dos casos a palavra final foi a de Dilma Rousseff. Nos três ela
foi a de sua neurose.

A \textit{mãe do PAC}, de Lula, deveria tornar-se, por seu próprio desejo, uma arcaizante
\textit{mãe de todos}, o que ela só conseguiu expressar frente a sua própria equipe, com
muitos desencontros. E esta era uma fórmula política e psíquica muito primitiva,
apenas inviável em uma democracia plena de forças contraditórias. O contraste
absoluto com o mundo do tipo de controle da política \textit{por sedução}, próprio de
Lula, é realmente espantoso, nos levando a pergunta se alguma vez houve de fato
algo em comum entre estes dois, homem e mulher, políticos de esquerda. Teria o
impulso obsessivo e controlador de Dilma Rousseff, de tecnocrata e matriarca, um
dia servido à organização psíquica da própria dispersão do homem político
verdadeiro que foi Lula – em uma reedição da \textit{imago} de Dona Lindu – e, deste
modo, ele pensou que ela faria tão bem ao Brasil quanto fez a ele, pessoalmente?
É difícil, até mesmo para um analista, acreditarmos que motivos psicanalíticos
tão prosaicos e tradicionais possam ter tamanho impacto público e histórico.

Os elementos que também se revelam fortemente na personalidade da
Presidente são, como venho sustentando, eles próprios políticos: por
cisão e ilegitimidade no trato com o governo -- na sua parte petista ou
não -- por arrogância simplória e gosto comezinho pelo poder no trato
com as forças externas ao governo, este \emph{modo de lutar} poderia
levar ao isolamento político, que em algum momento se tornou extremo.
Dilma funcionou na política simplesmente como uma \emph{anti-principe}
maquiavélico, ou seja, permanentemente incapaz de produzir uma ação
política, ou de linguagem, que \emph{aumentasse o seu próprio poder} e
que, também, aumentasse a integração de sua comunidade.

O único momento de bom humor legítimo, e sua inteligência, de Dilma
Rousseff -- quando ela teria dito que \emph{era uma mulher muito dura,
cercada de homens muito fofos} -- revela a crise de todos contra todos
no coração do terceiro governo petista. Esta crise pertence à política
brasileira, pertence ao \versal{PT}, e revela a inexistência real de medida
pública unificadora entre os seus homens. Mas ela se torna inteiramente
\emph{do governo} quando levada por uma política que reage a ela como a
Presidente sempre reagiu.

  \section{Que horas são?}\label{que-horas-suxe3o}

Quando se vive, em um estado de direito, em meio a denúncias constantes
de desvios de várias centenas de milhões de dólares de empresas
públicas, cuja culpa já foi assumida por vários réus em juízo; e quando
a empresa assaltada é a Petrobras; e quando milhares de pessoas, em sua
maioria das classes altas brasileiras, vão às ruas mais de uma vez pedir
o impedimento da Presidente; quando discursos públicos de governo são
encobertos com o som de panelas batendo, e quando o mesmo público, que
se levanta indignado por um sistema de corrupção, convive bem com outro
e com quem pede o retorno de alguma ditadura no Brasil; quando, após
anos de disseminação desta linguagem, ouvimos aos gritos que petistas
devem ir para Cuba -- no mesmo momento em que os Estados Unidos reabrem
relações com Cuba -- quando Ronaldo Caiado -- alguém ainda se lembra
quem ele é? -- tenta se tornar representante das ruas, e pede a extinção
do Partido dos Trabalhadores por corrupção reiterada; quando, da noite
para o dia, e sem manifestação do governo, o Congresso libera votações
da redução de maioridade penal e de uma quase ilimitada terceirização,
no país dos direitos trabalhistas varguistas que, entre outras coisas,
foram responsáveis pela criação dos sindicatos que deram origem ao \versal{PT};
quando papais e mamães ficam felizes com as fotografias de seus filhos
abraçados a policiais militares durante manifestações na Avenida
Paulista, em uma época em que a polícia brasileira é denunciada como uma
das que mais mata no mundo, especialmente jovens pobres e negros; quando
o espaço público da política imaginada se encontra em tal momento de
radicalização, de tensão e de esgarçamento dos sentidos, a favor de uma
difusa nova direita, em que cidadania parece ser apenas a garantia de
todos se desentenderem, bem como o evidente direito da grosseria
brasileira de se expressar nas ruas como política, talvez, então, nesta
hora histórica de meio transe, seja difícil -- para muitos dos que estão
excitados, ou correndo risco iminente de prisão, ou movidos pelos
interesses mais baixos de ódio e de vingança (de classe?) -- pensar com
processos de sentido mais amplos, que, todavia, nem sempre são meramente
simbólicos.

Mesmo que a tendência geral das ruas seja a do mais verdadeiro
embaralhamento das cartas, podemos lembrar alguns pontos históricos que,
na medida mínima do que é o pensamento, habitam os gritos mais gerais, e
o som estridente de panelas batendo forte ao redor. E outros ainda que,
embora bem importantes, não habitam de nenhum modo esta nova experiência
política, radical conformista.

De todo modo, a nova tecnologia de organização de uma ampla nova
direita, com seus textos abertos a todos os arcaísmos imagináveis, é ela
própria a sinalização de uma crise mais profunda: a das estruturas de
enunciação de alguma real perspectiva à esquerda neste campo. 
Com a acelerada falência do \versal{PT} tanto na política, quanto nas ruas, 
de onde uma voz crítica no Brasil poderá voltar a se tornar
um dia pública?
E, se em 2013 o Movimento Passe Livre conseguiu congelar
em vinte centavos o aumento das passagens de ônibus, quem vai conseguir
congelar a nova fúria organizada à direita, antissocial por tradição,
quando ela exigir, com terceirização do trabalho, todo tipo de corte nos
mínimos benefícios sociais brasileiros? O que observamos, no momento,
como sustentação ideológica do movimento, é que esta nova direita visa a
banir do espaço público a ideia de que qualquer reparação ao andamento
infernal do mercado no Brasil possa ser desejada.

  \section{A esquerda contra os
  bancos}\label{a-esquerda-contra-os-bancos}

Até onde pude acompanhar e compreender o processo da crise do governo
petista de Dilma Rousseff me parece necessário relembrarmos os pontos de
ruptura que antecederam e originaram todo o processo de falência
política do governo -- e de transbordamento social da oposição ao
governo -- que se configurou com força no início de 2015. Em algum
momento, por volta do ano de 2012, o terceiro governo petista começou a
produzir tensão e cisão de interesses que, em ondas
sucessivas de desgastes, tornou-se a base do imenso racha, de caráter
social, que expressou a posição classista de um novo tipo de paixão
política à direita, que se abateu de modo feroz sobre o início do quarto
governo petista.

De fato, esta longa sucessão, para o Brasil, de governos petistas – sob o
trabalho constante de uma oposição que se organizou como texto, como grupo e
como conjunto de interesses concretos ao menos desde 2012 – também faz parte dos
elementos que constituem a crise.
Após anos de desgastes, passado o tempo sobre um conjunto de mazelas de governo
que pouco se alteravam – a presença constante do sistema político em 
\emph{estado de corrupção} – e mantendo-se o poder, apesar de toda crítica, a oposição social
antipetista acabou por descobrir os mecanismos de protesto direto, imediatos e
de choque, que buscam reduzir muito a margem de manobra do discurso político do
governo, se é que, a esta altura, ainda há algum.

O que ocorreu, em meados de 2012, quando se produziu o primeiro discurso de
oposição organizado, com base social forte, que quebrou o relativo pacto
de interesses acomodados realizado por Lula durante seus dois governos?
Com o pacto que chegou ao auge em 2010 após a aprovação recorde de seu governo
e a eleição de sua sucessora, neófita no jogo do poder?

Uma importante nova onda de baixa dos juros bancários brasileiros,
promovida pelo governo, chegou ao seu ponto mais baixo, e gradualmente, disparou o
discurso e a organização social de interesses contra a política governista. A
taxa referencial dos juros brasileiros, a Selic, caiu continuamente, em
uma série histórica incomum. De agosto de 2011, quando alcançava 12,5\%,
até novembro de 2012, quando chegou a 7,25\%, a taxa atingiu de fato o
menor nível no Brasil em todos os tempos. Além disto, para relembrarmos
a persistência do governo naquela política, durante seis meses, de
novembro de 2012 a abril de 2013, a taxa foi mantida exatamente no
patamar de 7,25\%, baixo para o Brasil.

Para termos uma ideia do que isto significou, podemos lembrar que a
taxa Selic mínima alcançada no período Fernando Henrique/Armínio Fraga,
em fevereiro de 2001, foi de 15,25\%… No período Lula o ponto mais baixo
foi de 8,75\%, em julho de 2009. A média da taxa de juros dos anos Lula
foi de 13,79\%. E a de Fernando Henrique, fabulosos 26,7\%…

Deste modo, o governo de Dilma Rousseff afirmava com muita força uma
concepção de política econômica, e modo de funcionar, que o destacava da
tradição de facilitação dos interesses bancários, muito própria de nossa
democracia. O governo pretendeu de fato \emph{governar} o sistema geral
do dinheiro, e se pensava politicamente forte para tanto. Além de baixar
os juros gerais da economia, ele dirigiu os bancos públicos, Caixa
Econômica Federal e Banco do Brasil, para a queda real dos juros de
operação de créditos diretos ao consumidor, e dentre eles, até mesmo o
da tradicional extorsão cotidiana do cheque especial. E, exatamente
neste momento, interessante e interessadamente, emergiu com força a
oposição a este modo do terceiro governo petista de entender as coisas
do Brasil.

Foram muitos os tipos de discursos que então se apresentaram, e que
minavam e negavam a possibilidade da política de juros, que visava ao
desenvolvimento interno da economia, sustentada ao longo de quase dois
anos pelo governo, dar certo algum dia. Surgiu um mundo de profecias
sucessivas de uma crise econômica ruinosa que se abateria sobre o
Brasil, mas que, também no período, no mundo da vida e do pleno emprego
do comprometimento econômico petista, sempre se adiava.

Em abril de 2012, quando o processo de queda dos juros ainda estava
longe do auge, Roberto Luiz Troster, um ex-economista chefe da Febraban,
revelou a posição do setor em relação à política:

\begin{quote}
``O governo está determinado a baixar as taxas de juros bancárias. Para
tanto, Banco do Brasil, Caixa e Ministério da Fazenda estariam
preparando medidas para reduzir o custo do financiamento. O objetivo
seria fazer as duas instituições ofertarem linhas de cheque especial, de
aquisição de bens e de crédito pessoal a 2\% ao mês. Com isso, forçariam
as demais a emprestar mais barato. A ação do governo induziria a uma
eficiência maior do sistema financeiro, compatível com sua sofisticação.
Com isso, a inadimplência diminuiria, o consumo e o investimento seriam
estimulados, em especial das pequenas e médias empresas. Mas isso é
inviável. Lamentavelmente, da maneira que está sendo lutada, é uma
batalha perdida. Não é por falta de boa vontade ou de capacidade dos
envolvidos. Sem subsídios ou prejuízos, não é possível. Os grandes
bancos no Brasil não conseguem emprestar ao consumidor nesse patamar de
taxas. Basta analisar seus balanços e verificar que as margens almejadas
seriam deficitárias.''\footnote{``Crédito a 2\% ao mês? Não vai dar
  certo'', Folha de S. Paulo, 6/4/2012.}
\end{quote}

A diferença é bem grande do espírito pacificado da declaração do
banqueiro Olavo Setúbal, feita à mesma Folha de S.\,Paulo em agosto de
2006, às vésperas da segunda eleição de Lula: ``Lula está para ganhar a
eleição e o mercado está tranquilo. Não tem diferença do ponto de vista
do modelo econômico. Eu acho que a eleição do Lula ou do Alckmin é
igual.''

Este primeiro discurso contra a política de juros dilmista tinha o
mérito especial de ir ao ponto, de dizer aquilo que a partir de então as
demais construções de razões negativas passariam a esconder. Ele era
bastante claro: os bancos passariam a trabalhar com margens de lucros
que consideravam insuficientes, e assim, eles não aceitavam a política.
Quando os juros nos \versal{EUA} de Barack Obama estavam girando entre 0 e
0,25\%, para estimular a economia local à sair da crise, juros de 7,25\%
ao ano no Brasil eram inaceitáveis… Certo, se lembrarmos que nosso
sistema financeiro chegou a trabalhar durante anos com médias de
26,7\%… O governo comprara uma briga limite, com agentes sociais
especiais, setores muito organizados e influentes do Capital nacional.

E, neste ponto, não sabemos dizer, é difícil saber, se a política da
ruptura anunciada se produziu pela fome dos interesses financeiros
confrontados ou pela proverbial inabilidade política da Presidente em
conduzir processos de conflito, visando a alguma meta política estável --
inabilidade cifrada na enigmática frase de Troster, ``da maneira que a
luta esta sendo lutada…''? -- bem ao contrário de seu antecessor. O
fato é que, pela primeira vez em dez anos da meio surpreendente política
de \emph{capitalismo social} da esquerda que chegara ao governo em 2003,
surgia uma oposição consciente, interessada e com força social real
contra, capaz de, gradualmente, abalar em profundidade o projeto geral
petista.

A partir daí os discursos de confronto de modelos, de disputa de leitura
da vida econômica e social brasileira, se sucederam e se intensificaram.
As percepções técnicas fiscais, pró-diminuição da atividade econômica do
governo, ocuparam gradualmente cada vez mais espaço no sistema da
comunicação pública, e tornaram-se cada vez mais o consenso de uma nova
opinião política que se punha nitidamente contra o governo. Ao mesmo
tempo, e no mesmo movimento, os investimentos reais na economia
escasseavam. Assim, acumulavam-se com facilidade todas as críticas ao
ponto. Além de serem impossíveis para os bancos, como com singeleza
espantosa declarou o homem da Febraban, \emph{o capital nacional não
faria a inversão em mercado produtivo interno} que o governo almejava,
ele não investiria de nenhum modo, porque os parâmetros da economia
estavam manipulados, maquiados e subsidiados…

Além disso, os juros baixos apenas emprestavam dinheiro fácil a quem,
logo mais, não poderia pagar, o que colocava o sistema inteiro ainda
mais em risco… Além disso, o processo era apenas inflacionário, e não
continha, apesar do virtual pleno emprego continuado no país, nada de
virtuoso…

Qualquer argumento, mesmo sendo eles contraditórios, era válido para
desmobilizar a política e deslegitimar um governo que deixou de ser de
interesse. O governo Dilma Rousseff criara, finalmente, após dez anos de
predomínio petista, a sua oposição real. E, talvez, a mais forte
oposição que ele poderia esperar, a oposição direta do \emph{capital
financeiro}, e seu pacto pelo rentismo, sujeito vitorioso de todos
processos do capitalismo à brasileira, com acesso permanente, franqueado
e \emph{vip} a todo meio de comunicação. Este grande discurso, do desejo
deste ator das coisas do dinheiro entre nós, estruturaria gradualmente a
base de uma alternativa política ao governo petista, até então
inexistente.

Na disputa real pelos juros, pela primeira vez, o governo petista
\emph{deixou de ser o governo dos bancos brasileiros}, que passaram a
investir de novo na busca de um próprio mundo político de interesse. O
que não necessitavam fazer desde a chegada de Lula ao poder.

  \section{A esquerda contra a
  esquerda}\label{a-esquerda-contra-a-esquerda}

Por outro lado, em outra direção, em junho de 2013 o governo sofreu um
imenso revés, até então inconcebível e inexistente em qualquer quadro de
ação política imaginável. As ruas falaram alto contra os limites sociais
e a adaptação propagandística a uma realidade imaginada, que o governo
recebera do modo de Lula gerir a vida simbólica brasileira.

Da noite para o dia, longas e sérias crises acumuladas, do transporte
público nas grandes cidades brasileiras, da experiência coletiva da
polícia incompetente e antissocial, herança não criticada da ditadura
militar, da péssima qualidade dos serviços públicos no Brasil, todas as
mazelas de uma vida falsamente pacificada e redimida na lógica das
seduções muito propagandeadas do lulismo, entravam em crise imediata e
urgente e movimentavam uma nova paixão política, do desejo de política
direta, nos corpos vivos de jovens, há muito não vista no Brasil.

O Brasil entrava em fase, da noite para o dia, e apesar da tinta
imaginária lulista sobre a realidade nacional, com as crises populares
frente à representação da política que eclodiram, no pós-2008 da quebra
do sistema de Wall Street, em várias partes do mundo: no occupy
americano, nas manifestações espanholas, no mundo árabe africano, na
Turquia, na Grécia…

Todo o esforço propagandístico, com algo de verdadeiro e algo de falso,
do governo lulo-petista de mais de dez anos, aparecia de repente como
muito deficitário em relação às aspirações de cidadania qualificada da
vida urbana brasileira. Aos trocos do bolsa família para miseráveis,
política social limite, ao esforço de aumento mínimo de salário e do
crédito, para o avanço do consumo dos pobres no Brasil e à política de
atração do mercado do espetáculo mundial para o Brasil -- com Copa do
Mundo e Olimpíada passando a gerir investimentos, ações públicas e
legislação locais -- se contrapunha nas ruas, como crítica à esquerda ao
governo, a ruína da educação, da saúde, do transporte e da polícia. E,
enfim, da própria política.

E, no caso do heroico, jovem e amplo Movimento Passe Livre, se
anunciavam duas dimensões novas para a política, no quadro do populismo
de mercado lulista: o desejo de reformas mais radicalmente democráticas,
como a socialização do direito à mobilidade nas cidades do Brasil, e, a
novidade real imediata, a possibilidade de uma ação política
independente da vida oficial chegar a resultados concretos. O
desrecalque utópico das sociedades de classes, apesar dos mecanismos
cada vez mais sofisticados de assimilação e controle, é sempre um
espectro que ronda concretamente o mundo do mercado.

Duas perspectivas críticas muito aguçadas viram um horizonte mais amplo
no processo histórico da crise popular, nas ruas, frente ao pacto
interno de concertação capitalista da nação esgarçada:

\begin{quote}
``Até o próximo \emph{round} quando outros atores finalmente entrarem em
cena, saberemos se as jornadas de junho começaram de fato a desmanchar o
consenso entre `direita' e `esquerda' acerca do \emph{modus operandi} do
capitalismo no Brasil. Há vinte anos o país se tornou uma tremenda
fábrica de consentimento, todos empenhados em se deixar esfolar com
fervor. Batemos no teto? É o que a derrapagem histórica que detonou todo
o processo sugere. Pela primeira vez a violência que restou da ditadura
-- e a democracia aprimorou -- aprisionando a política no aparato
judiciário-policial, por algum motivo não funcionou. Um limiar
certamente foi transposto. Resta saber qual, e logo.''

\asterisc

``Em duas semanas o Brasil que diziam que havia dado certo -- que
derrubou a inflação, incluiu os excluídos, está acabando com a pobreza
extrema e é um exemplo internacional -- foi substituído por um outro
país, em que o transporte popular, a educação e a saúde públicas são um
desastre e cuja classe política é uma vergonha, sem falar na corrupção.
Qual das duas versões está certa? É claro que todos estes defeitos já
existiam antes, mas eles não pareciam o principal; e é claro que aqueles
méritos do Brasil continuam a existir, mas parece que já não dão a
tônica. O espírito crítico, que esteve fora de moda, para não dizer fora
de pauta, teve agora a oportunidade de renascer. A energia dos protestos
recentes, de cuja dimensão popular ainda sabemos pouco, suspendeu o véu
e reequilibrou o jogo. Talvez ela devolva a nossa cultura o senso de
realidade e o nervo crítico.''\footnote{A primeira citação é de Paulo
  Eduardo Arantes, a segunda de Roberto Schwarz. Duas perspectivas em
  que há semelhanças amplas, mas, também, alguma diferença significativa
  em relação à avaliação do \emph{sentido} da experiência histórica
  brasileira da redemocratização. Em \emph{Cidades rebeldes}, São Paulo:
  Boitempo, 2013. À época também escrevi, olhando o movimento de uma
  perspectiva de esperança de esquerda: ``Em termos históricos mais
  amplos, o que se anuncia nas ruas é o esgotamento do período de
  hegemonia do pacto social realizado pela política de Lula, incluindo
  aí o seu corpo, centrado na inclusão pelo consumo de superfície. Uma
  nova ordem crítica da política oficial, e sua distância da vida, e uma
  inédita crítica de massa à \emph{corrupção do espetáculo} -- visando
  aos gastos antissociais da Copa do Mundo no Brasil -- são também
  importantes avanços simbólicos, novas marcas políticas
  \emph{investidas}, que o movimento produziu. São três os principais
  \emph{significantes} que emergem da prática política coletiva:
  \emph{sem partidos} -- o que também quer dizer sem o \versal{PT} -- \emph{sem
  violência} e, ao redor do processo, \emph{ninguém está entendendo
  nada}. Em conjunto com o prazer do reconhecimento de um nível alto de
  \emph{esclarecimento das massas} no Brasil e da retomada do valor
  social da solidariedade.''; ``As manifestações e o direito à
  política'', Folha de S. Paulo, 24/06/2013.}
\end{quote}

O movimento social dos jovens independentes de esquerda pensava em
valores amplos e utópicos, mas perfeitamente possíveis, ao mesmo tempo
que ocupava o espaço real deixado pela alienação do Estado e da política
oficial em relação ao mundo da vida. Uma nova prática social, à
esquerda, anunciava a perda de contato do Partido dos Trabalhadores com
as forças que durante mais de trinta anos ele quis representar, e soube
integrar, para a hegemonia do grande projeto de Lula.

Estes dois grandes momentos e movimentos, o do abandono do dinheiro
financeiro do projeto da esquerda petista, e o da crise política da
perda das ruas para demandas sociais legitimas, que o governo não podia
responder, criaram a tensão original por onde o governo de Dilma
Rousseff começou a perder força. De um lado, o capital local se afastava
do projeto, começava a criticá-lo duramente no espaço da mídia pública,
e passava a buscar nova alternativa de interesse ao poder petista,
sustentando concretamente o movimento da forte oposição que quase
venceria a eleição de 2014. De outro, movimentos sociais à esquerda
escancaravam a distância que o partido mantinha da vida real das mazelas
brasileiras, propunham políticas de cunho sociais mais fortes e
encenavam a política independente, de ação direta, como perda de
influência do governo sobre as suas próprias bases tradicionais.

Definitivamente, para o bem e para o mal, Dilma Rousseff há muito não
\emph{era um poste} de Lula, e, no seu estilo Geisel de ser, muito
menos, era uma continuidade da ação política que emanava dele.

  \section{Anticomunismo, antipetismo}\label{anticomunismo-antipetismo}

Estas tensões políticas, clivagens e afastamentos sociais reais do
governo de Dilma Rousseff foram a base da convocação de um outro tipo de
agente social, que acabou por ser a fera de ataque mais dura, organizada
e eficaz, para a corrosão atual da mística petista. Com o realinhamento
gradual e real do grande capital contra o governo, \emph{o homem
conservador médio}, antipetista por tradição e anticomunista por
natureza arcaica brasileira mais antiga -- um homem de adesão ao poder
por fantasia de proteção \emph{patriarcal e agregada}, fruto familiar do
atraso brasileiro no processo da produção social moderna -- pode entrar
em cena como força política real, deixando de expressar privadamente um
mero ressentimento rixoso, carregado de contradições, contra o relativo
sucesso do governo lulo-petista, que jamais pode ser verdadeiramente
compreendido por ele.

Com as eleições, e o apoio senhoril assegurador do grande dinheiro, que
voltava a ser genericamente antipetista, este povo se manifestou em
massa. Com a bomba atômica da corrupção na Petrobras revelada,
explodindo no colo da Presidente logo após a reeleição -- a verdadeira
ficha do desequilíbrio político final -- esta camada média, que havia se
organizado ao redor de um candidato e que não se conformara com a sua
derrota, ganhou o instrumento definitivo, agora de fato \emph{real},
que, junto com a sua própria nova organização, de produção midiática de
espetáculo de massas, e de muita estratégia na internet, gerou a nova
paixão política conservadora pósmoderna brasileira. O desequilíbrio mais
profundo da política no capitalismo de consenso geral brasileiro,
indicado acima por Paulo Arantes, tendia a se desequilibrar fortemente
para a direita, \emph{nova velha}.

Assim, antipetistas indignados com a corrupção do outro, e
anticomunistas do nada, tomaram as ruas para produzir o texto para os
grandes conglomerados de mídia nacionais repercutirem, o que ocorreu, em
tempo real. Estas forças \emph{herdaram as ruas} a partir dos levantes,
originalmente críticos ao governo, mas à esquerda, ocorridos em 2013, se
apropriando da legitimidade política e simbólica do que era um outro
movimento.

Embora esvaziado em todo o mundo, e particularmente no modo de conceber
o poder da até ontem bem sucedida esquerda democrática brasileira, a já
tardia ideia de ``comunismo'' parece ainda ter uma vigência imaginária
importante no Brasil, e está bem presente, surpreendentemente, no fundo
da ação na rua desta grande fração das classes altas brasileiras. Onde
as coisas são assim, pode-se afirmar com alguma certeza um fracasso
do vínculo entre pensamento e política.

Construção que vem de bem longe, ponto de apoio e ideia central para a
instauração de duas ditaduras \emph{parafascistas} no difícil século \versal{XX}
brasileiro, foco de uma guerra mundial pela hegemonia de Impérios, o
anticomunismo sobrevive magicamente no Brasil de hoje como uma espécie
de imagem de desejo, para a grande simplificação interessada da política
que ele de fato realiza. Ele mantém o discurso político em um polo muito
tenso e extremo de negatividade à qualquer realização democrática ou
popular de governo; ou melhor, ele é contra qualquer realização que
desvie a posse imaginária do Estado de seus senhores, imaginários, de
direito.

Para antipetistas, movimento de desfaçatez do velho anticomunismo, basta
atribuir ao governo o epíteto de estalinista, ou bolivariano -- e gritar
nas ruas que `aqui não é a Venezuela', como se algum dia o Brasil o
tenha sido -- para poder se livrar de explicar todo o sentido real da
política brasileira. Trata-se de um sortilégio, da redução da política
ao maniqueísmo interessado mais simples, na esperança de desfechos já há
muito impossíveis, do tipo guerra fria.

A dinâmica democrática e viva entre as classes e o governo é
transformada deste modo em um gesto de desejo imediato, em uma luta
imaginária limite, contra os comunistas inexistentes. E, me parece, isto
apenas quer dizer que o governo deve ser derrotado \emph{in extremis}. O
anticomunismo é estratégia extremada -- ancorado no arcaico liberalismo
conservador brasileiro, com fumos de fidalguia, as famosas raízes do
Brasil, de origem ibérica e escravocrata -- de resgatar o governo de
compromissos populares quaisquer, mesmo quando estes compromissos, como
no caso dos governos Lula e Dilma, sejam de fato os da inserção de
massas no mercado de consumo e de trabalho, evidentemente pró-mercado,
capitalista.

E, de fato, é necessária uma fantasmagoria limite, exatamente por isso:
foi o governo de esquerda que deu uma certa solução política para o
avanço capitalista bem paralisado no Brasil do neoliberalismo periférico
dos anos 1990, dirigido pela grande elite econômica nacional. Bem ao
contrário da alucinose dos homens que ainda usam os termos próprios da
guerra fria, como se sabe, o governo de esquerda dinamizou intensamente
o capitalismo de mercado interno brasileiro, alcançando de fato um
virtual estado de pleno emprego no Brasil.

A taxa de desemprego caiu sem parar durante os governos petistas, de
12,4\% em 2003 para 4,8\% em 2014, enquanto, de 2009 a 2014, nos Estados
Unidos, origem da crise mundial, ela oscilou de 10\% para 7\%, na Itália
ela foi de 7 para 13\%, na França de 8,5 para 10,2\% e na Espanha…, de
18 para 27\%; e por isso mesmo, nos valores hegemônicos de uma cultura
total de mercado, tal governo só poderia ser vencido se lhe fosse
projetado o velho desejo autoritário brasileiro, o mais puro
anticomunismo com toques de moralismo neo-udenista, que, mais uma vez,
nada tinha a ver com o caso.

Por isso, inimigos políticos paralisados pelo sucesso mais geral do
governo Lula foram revolver os porões psíquicos do passado: após a
vitória de Lula com Dilma, Fernando Henrique Cardoso propôs, de modo
envergonhado, mas convicto, que o \versal{PSDB} guinasse à direita e José Serra
utilizou-se abertamente de retórica anticomunista em sua campanha contra
Dilma Rousseff. Justo eles dois, um dia vítimas da prática de ódio
político com que agora flertavam. Essa linguagem já se tornara quase
óbvia na campanha de Aécio Neves, campanha derrotada, provavelmente,
pelos pobres empregados do Brasil de 2014.

Vejamos os termos sociológicos, e a janela de oportunidades, de Fernando
Henrique Cardoso, para esta guinada do partido, contra um discurso
político ``visando ao povão'', a favor do que chamou de \emph{novas
classes possuidoras}, que deveriam ter os próprios interesses aguçados
por uma nova política à direita; e a favor do acento do discurso
moralista de elite, que fatalmente encontraria a velha estratégia
retórica do anticomunismo brasileiro:

\begin{quote}
``Enquanto o \versal{PSDB} e seus aliados persistirem em disputar com o \versal{PT}
influência sobre os `movimentos sociais' ou o `povão', isto é, sobre as
massas carentes e pouco informadas, falarão sozinhos. Isto porque o
governo `aparelhou', cooptou com benesses e recursos as principais
centrais sindicais e os movimentos organizados da sociedade civil e
dispõe de mecanismos de concessão de benesses às massas carentes mais
eficazes do que a palavra dos oposicionistas, além da influência que
exerce na mídia com as verbas publicitárias. (…) Existe toda uma gama
de classes médias, de novas classes possuidoras (empresários de novo
tipo e mais jovens), de profissionais das atividades contemporâneas
ligadas à \versal{TI} (tecnologia da informação) e ao entretenimento, aos novos
serviços espalhados pelo Brasil afora, às quais se soma o que vem sendo
chamado sem muita precisão de `classe c' ou de nova classe média. Digo
imprecisamente porque a definição de classe social não se limita às
categorias de renda (a elas se somam educação, redes sociais de conexão,
prestígio social etc.), mas não para negar a extensão e a importância
do fenômeno. Pois bem, a imensa maioria destes grupos sem excluir as
camadas de trabalhadores urbanos já integrados ao mercado capitalista
está ausente do jogo político-partidário, mas não desconectada das redes
de internet, Facebook, YouTube, Twitter etc. É a estes que as oposições
devem dirigir suas mensagens prioritariamente, sobretudo no período
entre as eleições, quando os partidos falam para si mesmo, no Congresso
e nos governos. Se houver ousadia, os partidos de oposição podem
organizar-se pelos meios eletrônicos, dando vida não a diretórios
burocráticos, mas a debates verdadeiros sobre os temas de interesse
dessas camadas. (…) Seria erro fatal imaginar, por exemplo, que o
discurso `moralista' é coisa de elite à moda da antiga \versal{UDN}. A corrupção
continua a ter o repúdio não só das classes médias como de boa parte da
população. Na última campanha eleitoral, o momento de maior crescimento
da candidatura Serra e de aproximação aos resultados obtidos pela
candidata governista foi quando veio à tona o `episódio Erenice'. Mas é
preciso ter coragem de dar o nome aos bois e vincular a `falha moral' a
seus resultados práticos, negativos para a população. Mais ainda: é
preciso persistir, repetir a crítica, ao estilo do `beba Coca Cola' dos
publicitários. Não se trata de dar-nos por satisfeitos, à moda de
demonstrar um teorema e escrever `cqd', como queríamos demonstrar. Seres
humanos não atuam por motivos meramente racionais. Sem a teatralização
que leve à emoção, a crítica moralista ou outra qualquer cai no
vazio.''\footnote{``O papel da oposição'', Revista Interesse Nacional,
  no. 13, abril/junho 2011.}
\end{quote}

\versal{FHC} simplesmente sinalizou, em um discurso estranho e novo à leitura política
nacional, muito assemelhado aos cálculos sociais de marqueteiros americanos, a
brecha possível para a emergente \textit{tea partização} do espaço público da política
brasileira, um movimento apaixonado de busca de submissão extrema de tudo ao
mercado e sua estrita produtividade -- jacobinos do mercado -- que também animou,
em outro círculo do conservadorismo, o delírio arcaico do velho anticomunismo
brasileiro. Anticomunistas do nada, velhos autoritários anti-populares e novos
\textit{tea-partistas} em busca de um Estado estrito para a multiplicação de seus
negócios, iam de mãos dadas. E incluíam também na foto, feliz, pela primeira
vez como ator democrático, não por acaso, a problemática Polícia Militar
paulista. 

Também, no período de ascensão e queda petista, atacar com a máxima
retórica, isenta de responsabilidade, em jornais, blogs ou revistas, o
comunismo imaginado do governo, tornou-se um dos modos mais fáceis e
oportunos de ganhar dinheiro no mercado dos textos e das ideias no
Brasil. Era suficiente reproduzir a rede de ideias comuns e fixadas, com
sua linguagem agressiva, indignada artificial, que sustentassem todo dia
o mesmo curto circuito do pensamento. Simplificação espetacular e ponto
certo no imaginário autoritário, jornalistas, articulistas, programas de
\versal{tv} e de rádio e revistas inteiras passaram, durante anos, a ler as
atividades do governo do ponto de vista extremo, limitado, do
anticomunismo imaginário. Além de anacrônico, havia algo de
verdadeiramente preguiçoso neste processo mental político. Antigos
artistas, verdadeiros comunistas dos anos 1960 -- os nomes são
conhecidos de todos -- se prestavam a vender opiniões imediatas,
atacando faceiramente o aberto \emph{estalinismo} dos governos de Lula e
Dilma. Surgiram os muito duvidosos heróis intelectuais do gênero.

Embora a imprensa fosse absolutamente livre, a Polícia Federal, o
Ministério Público e a Justiça trabalhassem como jamais no Brasil, e
desde o segundo ano do governo Lula a cúpula petista estivesse sobre
processo criminal aberto e acabasse de fato inteira na cadeia, durante
anos homens muito inteligentes nos garantiam todos os dias nos jornais a
natureza ditatorial fixada -- alucinose -- do governo petista.

O delírio interessado, farsesco, não conhecia limite, uma vez que se
desobrigava radicalmente de checar realidades. O fato de, contrariando a
opinião garantida destes estranhos pensadores, sempre dada por certeza,
Lula não ter se aventurado por nem um segundo na busca de um terceiro
mandato, como era previsto -- bem ao contrário do comportamento de \versal{FHC}
quando na Presidência -- também não os sensibilizou para os compromissos
democráticos do Presidente petista. E gradualmente, se abria mais e mais
o espaço para este tipo de regressão, \emph{wishful thinking}, da
leitura da ordem da política, impingindo o delírio apolítico, trabalho
mágico obsessivo, como a medida real das coisas brasileiras.

No limite, chegamos a conviver cotidianamente, em grandes jornais, com
articulistas que atacavam qualquer ideia ou projeto progressista, de
interesse coletivo, solidário ou, até mesmo, apenas meramente humanista.
Os novos modernos anticomunistas liberais do mercado concentracionário
brasileiro tangenciavam o fascismo, um tipo muito próprio de fascismo de
consumo, como dizia Pasolini. Daí a emergência lógica de um discurso
final, atual, baseado no mesmo jogo grosseiro de redução da política, da
ideia apoteótica de extermínio definitivo do \versal{PT}…

O fato de o governo Dilma ser obrigado a convocar, algo contra a vontade,
uma Comissão Nacional da Verdade, após o Brasil, no apagar das luzes do
governo Lula, ao final de 2010, ser enfim condenado na Corte
Interamericana de Direitos Humanos da \versal{OEA}, também mobilizou a ira de
velhos torturadores aposentados, amigos e parentes de torturadores e
saudosos brasileiros de ditaduras de todos os tipos, que, em tal
panorama, puderam falar contra a tardia Comissão da Verdade da
democracia brasileira, e o governo, sem sofrerem nenhum constrangimento
de opinião pública, ou legal.

Como se sabe, tais homens bons foram cruelmente perseguidos pela sanha
revanchista dos comunistas derrotados, que haviam tomado o poder de
assalto em 2003 e, assim, estes homens bons estavam legitimados, pelos
próprios interesses, a retornarem ao ideário de 1970, época em que
torturavam, matavam e desapareciam com brasileiros… Era preciso manter
a paranoia alimentada.

Os anticomunistas, agentes reais de ditadura, foram convocados pela
mínima política reparatória forçada à esquerda, pois foram incomodados
em suas aposentadorias especiais e premiadas. Pela estratégia geral da
luta política contra o governo eles foram cinicamente tolerados.

Assim se produzia o campo extremo, algo delirante, em que a luta
democrática antipetista encontrava a velha tradição autoritária
brasileira. E, por isso, agora que o país, em seu neo-transe, se levanta
contra os comunistas inexistentes, em uma ritualização do ódio e da
ideologia, elegantes \textit{socialites} peessedebistas e novos empresários
\emph{teapartistas} convivem bem, nas ruas, fechando os olhos para o que
interessa, com bárbaros defensores de ditadura, homens que discursam
armados em cima de trios elétricos, clamando por intervenção militar
urgente no Brasil e sonhando com o voto em Jair Bolsonaro. Não por
acaso, em regime de farsa verdadeira, se vislumbrou nas passeatas de
março o semblante das velhas marchas conservadoras de 1964.

Assim, todo o campo dos anticomunistas do nada, incluindo elegantes
estadistas e cientistas sociais, prestou desserviço à qualificação do
debate público brasileiro para a vida contemporânea, que ainda é
seduzido e obrigado a pensar, por estes homens, regressivamente, com
parâmetros vencidos de mundo, construídos em 1959. Este campo também é
movido, em uma certa facção da elite que o anima, por uma verdadeira
política \emph{identitária de} \emph{classe}, cujo lastro organizador de
mundo é o ódio antipopular brasileiro. 

Tal grosseria imatura e interessada seria simplesmente inaceitável por
alguma vida política minimamente informada; se não se apoiasse em
espetaculares erros reais do governo, que talvez, imaginariamente,
entenda que a crítica às suas práticas graves seja apenas a ideia fixa
delirante do anticomunismo do nada, e não um gradual e verdadeiro
afastamento de suas bases políticas.

O anticomunismo atrasado brasileiro é regressão da política. Regressão
aos argumentos de força e redução da diferença, e implica gozos baixos,
do ódio que poderia se alçar ao sadismo, simplificação de toda vida
pública e social e do direito ao desprezo do destino da vida
popular. É uma política do direito ao ódio fixado, frente à vítima
escolhida.

Ele tende, como pode se observar facilmente no Brasil hoje, a reduzir a
linguagem mediada dos problemas ao gesto de força, na panela, ou no
corpo do inimigo.

  \section{Economia política: teoria
  tradicional}\label{economia-poluxedtica-teoria-tradicional}

Podemos sintetizar o resultado da crise econômica do governo Dilma, 
\emph{liquidado pela impossibilidade de manter sozinho, por subsídios internos a grupos
selecionados, a vida econômica do Brasil aquecida, e assim, empregando, em uma
economia global com viés claramente recessivo}. Isto se deu em uma época na qual,
por exemplo, a Europa e seu Capital se viram em crise aguda, sob os grandes
riscos da quebra da Grécia e a grande crise da invasão social dos miseráveis sem
destino periférico, africanos e árabes; quando se fragmentou e se multiplicou a
guerra americana no oriente médio, tendente à guerra perpétua; quando se
instaurou a importante crise geopolítica e econômica russa; e, principalmente,
para o Brasil, quando se deu a real desaceleração da economia Chinesa, que se
tornou uma constante. 

Esta síntese pode ser mais bem feita por alguém que, neste momento,
acredita \emph{nestes} parâmetros; ela pode ser dada nos termos finais
realizados por um especialista, um verdadeiro \emph{economista
político}, para o bem e para o mal. Trata-se de um ambíguo comentarista
que apoiou genericamente, mas positivamente, como um representante
\emph{agora} progressista da burguesia paulista a política econômica
ampla de populismo de mercado lulista -- e que, na sua origem de homem
político, foi um tecnocrata poderoso forte da ditadura militar
brasileira em seu auge --. E que, ao fim e ao cabo da degradação das
contas públicas, do fracasso do keynesianismo de subsídios da \emph{nova
matriz econômica dilmista} -- sem inversão ou criação de nenhum capital
produtivo -- que chegou ao seu esgotamento, repôs assim, da noite para o
dia, os velhos parâmetros liberais \emph{universais} de leitura
econômica, em um rápido retorno à ordem pré-petista, agora pós-petista:

\begin{quote}
``O ano 2014 foi horrível. Nele prevaleceu a `vontade' da reeleição a
qualquer custo. Ela era necessária para fechar o ciclo de uma geração de
domínio do Partido dos Trabalhadores, do qual emergiria,
definitivamente, o `nosso Brasil', como diz o seu Presidente. A `vontade
política' preteriu, assim, as mínimas condições impostas pelas
restrições físicas que mantêm um razoável equilíbrio econômico. Tivemos:
um déficit fiscal de 6,2\% do \versal{PIB} (contra 3,1\% em 2013); uma taxa de
inflação de 6,4\%, mas que escondeu os efeitos de preços controlados da
ordem de 3\% a 4\%; a relação Dívida Bruta/ \versal{PIB} aumentou em 6\% do \versal{PIB};
um déficit em conta corrente de \versal{US}\$ 104 bilhões (4,4\% do \versal{PIB}) e, por
fim, uma queda de 0,7\% do \versal{PIB} per capita.

Permanecendo no poder, o \versal{PT} acreditava que teria tempo de sobra para dar
a `volta por cima' e preparar-se para ganhar as eleições de 2018.

As provas materiais dessa hipótese são um relatório interno de 2013, da
Secretaria de Política Econômica do Ministério da Fazenda, que já
apontava que a velocidade de crescimento das despesas primárias do
governo era maior do que a da receita, que vinha sendo coberta com
receitas atípicas, isto é, não recorrentes. Chamava a atenção para a sua
aleatoriedade.

Outro relatório interno da mesma origem, de 2014, propunha `exatamente'
as medidas corretivas iniciais do `ajuste' fiscal que o Governo só
enviou ao Congresso depois de reeleito. O Ministério da Fazenda
imolou-se no altar da fúria de poder do \versal{PT}. Inventou a `nova matriz
econômica' para dar cobertura à irresponsabilidade política. Como me
ensinou meu velho avô, `quando alguém erra três vezes na mesma direção,
preste atenção, porque provavelmente ele está acertando'…

Houve uma trágica subestimação dos efeitos deletérios dessa estratégia.
Na tentativa de corrigir o estrago eleitoral a Presidente impôs-se uma
conversão comparável à de são Paulo na estrada de Damasco. Teria
funcionado se ela não tivesse, ao mesmo tempo, perdido a confiança dos
seus eleitores, o que tornou pior o que já estava ruim. Somou à crise
econômica uma crise política, como é frequente quando o Executivo perde
o seu protagonismo.''\footnote{Delfim Netto, ``É estrutural'', Folha de
  S. Paulo, 29/07/2015.}
\end{quote}

As súbitas conversões petistas, e seu novo \emph{beijar a cruz}, já não
convenciam mais, ou se tornaram redundantes e desnecessárias. E é a
certeza da sustentação desta perspectiva que deu as reais garantais
simbólicas para a aposta continuada da derrubada, nas ruas, do governo
reeleito em 2014.

  \section[Nova direita: Eduardo Cunha]{Nova direita: Eduardo
    Cunha}\label{nova-direita-eduardo-cunha}

Eduardo Cunha,\footnote{O que segue, a respeito de Eduardo Cunha, foi publicado
  anteriormente em O Estado de S. Paulo.} do \versal{PMDB} do Rio de Janeiro, assumiu a presidência da
Câmara dos Deputados em 1º de Fevereiro de 2015. Sua vitória, por 276
votos a 136, contra Arlindo Chinaglia, o candidato do governo, foi a
primeira das várias derrotas que a partir de então, em ritmo
vertiginoso, ele passara a promover no Congresso, contra temas, pautas e
princípios do governo petista de Dilma Rousseff. A organização pessoal
de Cunha, e de seus interesses conservadores amplos, imediatamente
ganhou nítido contraste com a dissolução geral da política petista, que
acontecia ao seu redor.

Apesar da vitória para a Presidência e de conquistar a maior bancada no
Congresso, o Partido dos Trabalhadores pareceu ter saído das urnas em
2014 simplesmente derrotado. A crise de corrupção na Petrobras --
envolvendo propinas do cartel que controlava a empresa endereçadas a \versal{PT},
\versal{PMDB} e \versal{PP} e a 17 políticos investigados, entre eles Eduardo Cunha e
Renan Calheiros, além de um senador do \versal{PSDB}… -- e o acirramento da
oposição que levou Aécio Neves a meros 3\% de distância da Presidente
reeleita, parecem ter marcado de maneira profundamente negativa o
espírito do novo governo. O desagaste total da sua política econômica,
que manteve o pleno emprego no Brasil, mas gastou todas as fichas
disponíveis para além do limite do equilíbrio fiscal e não conseguiu
promover crescimento no último ano e meio, levou o animo e a
auto-concepção do governo petista à lona. O governo só parecia saber
fazer política na plena posse do seu próprio modelo de economia -- uma
espécie de social-desenvolvimentismo, ou capitalismo social, se olharmos
daqui ou dali -- e ter de realizar cortes fortes nos gastos públicos, de
tipo neoliberal, desorientou definitivamente a bússola governista para o
próprio trabalho. Além disso, logo, a nova organização social à direita,
a nova paixão política à direita, prosseguiu sua feroz crítica ao
governo nas ruas, criando um fator de forte instabilidade que o \versal{PT} não
conhecia.

O quadro de fraqueza de governo, de falência de projeto, falta de
predomínio sobre a própria base, além da velha inapetência para a
política parlamentar petista -- uma agremiação que acabara viciada no
seu bonapartismo lulista --, não foi criado por Eduardo Cunha, mas é o
\emph{setting} que permitiu a força e o colorido da sua própria atuação
surpreendente. E desde o primeiro segundo ele soube ler com grande
acuidade a situação: ``Eu acho pouco provável o sucesso de uma
candidatura do \versal{PT} em qualquer disputa contra qualquer um, e não vejo
dentre os Deputados do \versal{PMDB} nenhuma vontade de apoiar uma candidatura do
\versal{PT} {[}à presidência da Câmara{]} (…) Até para a governabilidade é
melhor que se tenha alguém que seja representativo de uma maioria, que
tenha capacidade de discussão e que esteja completamente distante do
centro da polarização eleitoral que aconteceu.'' A respeito da
governabilidade, esta posição preliminar de Cunha era simplesmente uma
carta falsa.

Buscando legitimidade, como nome da maioria informe, em nome da sua
independência e como mediador, e virtual vencedor, da grande polarização
eleitoral de 2014, ele se elegeu Presidente da Câmara, contra um
movimento político ligeiro e desorgânico do governo. Imediatamente
declarou que só passando sobre seu cadáver o tema do aborto -- e o da
regulação da mídia -- seriam pautados na Câmara presidida por ele.
Aliás, de modo acintoso, e sinalizando claramente seu vínculo com um
grande sistema de poderes, ele também afirmou que ``regulação econômica
de mídia já existe no Brasil, você não pode ter mais de cinco geradoras
de televisão''… As declarações deixavam claro, no mundo da vida e das
grandes aspirações e perspectivas de poder, a posição de compromissos de
Cunha, e, ainda mais, a sua forte vontade pessoal de exercer ao máximo o
seu poder, empenhando simbolicamente o próprio corpo. Ele assumia a
responsabilidade por sua pauta conservadora, ainda mais à direita do
próprio grupo em geral informe, e voraz, de parlamentares que passou a
liderar, dar cérebro e coluna, conhecido historicamente como o
\emph{centrão}.

Sobre o aborto, a sua posição era tradicional. Homem ligado ao movimento
político de massas das Igrejas Evangélicas brasileiras -- mais
precisamente à Assembleia de Deus, ministério Madureira -- em 2011 ele
se tornou conhecido pela excentricidade e a desfaçatez de propor o
projeto de lei do ``dia nacional do orgulho hétero'', para defender a
``maioria discriminada'', segundo ele próprio. Quando, após a novela da
Globo \emph{Amor à vida} exibir em seu último capítulo o muito aguardado
beijo gay entre Félix e Nico, Cunha se manifestou prontamente e
assertivamente, como é sua característica, agora no Twitter: ``Estamos
vivendo a fase dos ataques, tais como a pressão gay, a dos maconheiros e
abortistas. O povo evangélico tem que se posicionar.'' Não se pode negar
que Cunha é um político em busca de ação. Ele se baseia na força social
real da massa de evangélicos, popular e moralista, e expressa para este
povo que o sustentou até agora uma espécie pós-moderna de amplo
populismo conservador, de fantasia de restauração, meio religiosa, meio
moral, meio midiática, que, como se tornou comum comentar a seu
respeito, não esconde o vínculo entre o desejo de ocupar espaço na
política, o caráter conservador e a ação muito afirmativa, autoritária.

Quanto aos grandes negócios, em grande escala, de novos monumentos à
ascensão da \emph{sua própria classe média}, Eduardo Cunha é também
explicitamente ativo. Além de se preocupar pessoalmente com a limitação
de cinco geradoras de televisão por cidadão brasileiro, assim que
assumiu a mesa diretora da Câmara ele deu início ao processo de
construção de um novo conjunto de prédios ligados ao Congresso, nos
quais estão previstos maiores gabinetes, estacionamento, um novo
auditório -- um novo plenário, mais amplo e para mais deputados,
desbancando o prédio tombado, patrimônio da arquitetura mundial, de
Oscar Niemayer? -- e um shopping de luxo para os congressistas e suas
mulheres. O valor total do empreendimento imobiliário, que envolverá,
mais uma vez, muitas das empreiteiras que investiram forte nas campanhas
dos próprios Deputados, está orçado em um bilhão de reais.

Os luxos e os mimos de Cunha aos seus pares deverão ser construídos no
período de maior aperto fiscal e orçamentário do país dos últimos 15
anos, que vai atingir em cheio a vida econômica e a classe trabalhadora
brasileira. O governo de Cunha para o Congresso se expressa na máxima
brasileira do político conservador, sob suspeição, que justifica e
legitima sua prática social meramente ao fazer construções de imenso
porte…

E as obras, agora, são gabinetes de luxo e shopping center particular,
no mundo dos chics entre si do parlamento brasileiro. Trata-se de um
sintoma mais que perfeito do tempo. E, além do negócio imobiliário
bilionário para o maior conforto da própria classe, há indícios
importantes, segundo promotores públicos, com declaração explícita do
delator Alberto Youssef na Operação Lava Jato, de Eduardo Cunha ter
recebido propinas e tentar chantagear a empresa japonesa Mitsui \& Co.
quando os pagamentos, ligados aos grandes desvios acontecidos na
Petrobras, cessaram. E, finalmente, em 16 de junho, Julio Camargo, homem
ligado à construtura Toyo Setal, preso e sob regime de delação premiada
da Operação Lava à Jato, declarou ao juiz Sérgio Moro:

\begin{quote}
``Tivemos um encontro. Deputado Eduardo Cunha, Fernando Soares {[}o
doleiro tido por homem de Cunha{]} e eu. {[}…{]} Deputado Eduardo
Cunha é conhecido como uma pessoa agressiva, mas confesso que comigo foi
extremamente amistoso, dizendo que ele não tinha nada pessoal contra
mim, mas que havia um débito meu com o Fernando do qual ele era
merecedor de \versal{US}\$ 5 milhões.''
\end{quote}

O Deputado, definitivamente, não é peixe pequeno.

Eduardo Cunha é, portanto, uma mistura de Severino Cavalcanti e Pastor
Marco Feliciano, que finalmente deu certo. Mais ativo, hábil com a
linguagem, sempre de prontidão e com imenso gosto pelo poder, mais
estruturado em sua bases, ele se tornou rapidamente, voando baixo sob a
ruína da política petista, o homem que a direita brasileira há muito
tempo não dispunha. Ele é de fato, sobre todos os aspectos, um
\emph{anti-Dilma Rousseff}. Uma verdadeira liderança, produtiva para os
próprios interesses de classe, com um aceno conservador nítido ao novo
público político produzido nas igrejas, nas rádios e nas televisões
evangélicas. Ele é egresso desta nova elite, popular, telerreligiosa,
empreendedora, ativa e pós-moderna.

Politicamente, em dois meses de direção da Câmara ele liderou duas
derrotas históricas, em velocidade de \emph{Blitzkrieg}, que sinalizaram
claramente a atual alienação política do governo dito de esquerda: a
aprovação na Comissão de Constituição e Justiça da proposta de
diminuição da maioridade penal no Brasil e a aprovação em plenário de
uma ampla lei de terceirização do trabalho no Brasil, produzindo a maior
derrota nas leis trabalhistas brasileiras desde a sua instauração por
Getúlio Vargas. Neste momento podemos dizer que o \versal{PT} simplesmente não
existia na política do país. Novamente, em uma frente, a pauta
conservadora visava a sociedade civil, e noutra, se articulava para os
maiores interesses empresarias, que aproveitaram o homem muito disposto
no Congresso e a falência do governo para tramitar da noite para o dia,
sem debate, um projeto de imenso impacto social, parado a 11 anos.
Nestes primeiros dias, Cunha valeu-se do vácuo de governo para
simplesmente governar o Brasil, desde o Congresso.

Com requintes de arbítrio -- pondo em votação novamente a matéria após
já ter sido derrotado -- Cunha buscou fazer a mesma coisa com a reforma
política, tentando antecipar-se à crise nacional que também promovera:
aprovar em velocidade extrema a reforma da sua preferência, sem
discussão. Pela primeira vez foi derrotado, em parte. Havia mais
interesses em jogo do que a sua liderança agressiva podia dar conta. Mas
garantiu o que lhe era principal: o atrelamento das campanhas políticas
brasileiras ao dinheiro das empresas.

Sua política acelerada e afirmativa, para os seus, que enfiou vários
gols no time do governo, perdido em campo, é também um modo de colocar os
interesses dos seus políticos denunciados na Lava Jato, entre eles ele
próprio e Renan Calheiros, no ataque; não apenas frente ao Planalto,
para quem pretendem transferir toda a responsabilidade das próprias
possíveis propinas, mas também, agora, frente ao Ministério Público e a
Justiça, que estão na mira dos canhões da Casa. Há algo de rei nu em
Cunha, e daí, também, tanta movimentação.

Por fim, no retorno do recesso parlamentar, após ter sido explicitamente
denunciado como corrupto ativo na Lava Jato, seu poder chegou ao máximo, já
apontando a desfaçatez nacional geral que se formou ao redor de sua atuação, já
há muito ilegítima: em uma única noite Cunha botou em votação e fez aprovar em
sua Câmara rompida com a Presidente as contas públicas do governo de Itamar
Franco, dos dois governos de Fernando Henrique Cardoso, dos dois governos Lula,
limpando a pauta do tema e abrindo condições formais para julgar as contas de
Dilma Rousseff – caso recusadas, politicamente, pelo Tribunal de Contas da União
–  e finalmente abrir o processo de impeachment contra ela. O casuísmo como
normalidade, a virtual ilegalidade das ações, a decisão unilateral em proveito
próprio e de seu grupo particular, nada faz com que haja reação pública às
práticas políticas de exceção do Deputado -- que se revelam assim de interesse
mais amplo -- verdadeiro conspirador à direita, por dentro do aparato
institucional da democracia.

E é apenas o elemento da virtual Justiça democrática, com a novidade
radical no panorama brasileiro que é o Juiz Sérgio Moro, o dado que
diferencia hoje a situação de um político como Eduardo Cunha de outros
políticos brasileiros da mesma estirpe, desde um Bernardo Pereira de
Vasconcelos, no Império, até um Carlos Lacerda, no pré-64.

  \section{Nova política:
  judiciário}\label{nova-poluxedtica-judiciuxe1rio}

``O juiz Sérgio Moro é um enigma''. Assim me falou um amigo que trabalha
no judiciário paulista.

Um enigma do Brasil contemporâneo. Muito foi dito sobre ele, em todas as
direções possíveis. Suas ações, ligadas à Polícia Federal e ao
Ministério Público do Paraná, fizeram parte de uma ofensiva eleitoral
peessedebista? Ele é um jacobino do judiciário, que busca levar a
judicialização da política a um novo patamar? Um homem moral, que
atingiu em cheio os vínculos promíscuos entre Capital e política no
Brasil? Um perverso, que constrange os direitos individuais, e cria
insegurança jurídica, para perseguir suas metas? Um homem de esquerda,
que pode prender arquimilionários no país da impunidade garantida? Tudo
isto foi dito dele, e ainda mais será.

O fato é que o juiz se preparou, técnica e politicamente, para a
operação judiciária incomum que dirige. Em meados dos anos 2000 ele
conseguiu relativamente pouco, em termos políticos e de justiça efetiva,
quando da operação \emph{Big Brother}, do processo do Banestado do
Paraná, um banco que lavou dezenas de bilhões de dólares enviados ao
exterior, na era das privatizações de Fernando Henrique Cardoso. E que,
posteriormente, foi privatizado, ainda no governo tucano.\footnote{``Durante
  o ano de 2003, uma força-tarefa do \versal{MPF} e da \versal{PF} investigou um esquema
  de evasão de divisas para paraísos fiscais operado por meio de contas
  \versal{CC}5 (específicas para transações de cambio tanto para pessoas físicas,
  quanto para pessoas jurídicas) do Banco do Estado do Paraná
  (Banestado). Essas contas \versal{CC}5, na época, eram de livre movimentação
  entre o Banestado, bancos estrangeiros e empresas de fachada, conforme
  revelaram as investigações. Ao todo, o esquema, estimavam os
  procuradores, pode ter movimentado aproximadamente \versal{US}\$ 28,1 bilhões
  durante os anos 1990. Durante o caso Banestado, mais de 1,1 mil contas
  foram investigadas, havendo o bloqueio de R\$ 380 milhões no Brasil e
  mais R\$ 34,7 milhões no exterior. As investigações foram encerradas,
  oficialmente, em setembro de 2007. Ao todo, o escândalo resultou na
  oferta de 95 denúncias classificadas de alta complexidade, entre elas
  uma contra Alberto Youssef. Houve, ao todo, 684 pessoas denunciadas por
  mais diversos crimes como lavagem de dinheiro, evasão de divisas,
  corrupção ativa e passiva, entre outros. O escândalo resultou em 97
  condenações, quase a grande maioria gerou sentenças relacionadas à
  prestação de serviços à comunidade. Apesar do grande número de
  condenações, as pessoas consideradas líderes do esquema foram
  beneficiadas pela prescrição dos crimes. De 14 ex-gerentes e
  ex-diretores do Banestado, sete tiveram suas penas extintas por
  decisão do Superior Tribunal de Justiça (\versal{STJ}) em 2013. Outros três
  tiveram suas penas parcialmente prescritas e ainda cumprirão cinco
  anos de prisão. Dois outros ex-diretores assinaram acordo de delação
  premiada e já cumpriram pena. Apenas um está cumprindo pena de
  prestação de serviços comunitários.'' Em
  http:goo.gl/i1B1To.
  O Banestado produziu a sua lavanderia internacional de dólares no auge
  do período das privatizações do governo Fernando Henrique Cardoso, e
  foi ele próprio privatizado, vendido ao Itaú, de modo que o Estado
  incorporou toda a dívida do banco.} \emph{Crime de lavagem de
dinheiro} é o título do livro que publicou em 2010, baseado no processo
do Banestado. Antes da Operação Lava Jato, ele ia de bicicleta para o
tribunal. Também é mestre e doutor em Direito do Estado, pela
Universidade Federal do Paraná e ainda foi o juiz mais votado na Justiça
Federal para a indicação à vaga de Joaquim Barbosa, no Supremo Tribunal Federal,
o que revela reconhecimento efetivo de seus pares. Seu estudo sobre a operação
mãos limpas demonstra sua posição teórica ampliada, entre criminal e política,
para o manejo de casos de grande corrupção, estrutural, envolvendo o Estado:

\begin{quote}
``A deslegitimação do sistema {[}\emph{político}{]} foi ainda agravada
com o início das prisões e a divulgação de casos de corrupção. A
deslegitimação, ao mesmo tempo em que tornava possível a ação judicial,
era por ela alimentada: A deslegitimação da classe política propiciou um
ímpeto às investigações de corrupção e os resultados desta fortaleceram
o processo de deslegitimação. Conseqüentemente, as investigações
judiciais dos crimes contra a Administração Pública espalharam-se como
fogo selvagem, desnudando inclusive a compra e venda de votos e as
relações orgânicas entre certos políticos e o crime organizado. As
investigações \textit{mani pulite} minaram a autoridade dos chefes políticos --
como Arnaldo Forlani e Bettino Craxi, líderes do \versal{DC} e do \versal{PCI} -- e os
mais influentes centros de poder, cortando sua capacidade de punir
aqueles que quebravam o pacto do silêncio.

O processo de deslegitimação foi essencial para a própria continuidade
da operação \textit{mani pulite}. Não faltaram tentativas do poder político
para interrompê-la. Por exemplo, o governo do primeiro-ministro Giuliano
Amato tentou, em março de 1993 e por decreto legislativo,
descriminalizar a realização de doações ilegais para partidos políticos.
A reação negativa da opinião pública, com greves escolares e passeatas
estudantis, foi essencial para a rejeição da medida legislativa. Da
mesma forma, quando o Parlamento italiano, em abril de 1993, recusou
parcialmente autorização para que Bettino Craxi fosse processado
criminalmente, houve intensa reação da opinião pública. Um dos protestos
populares assumiu ares violentos. Uma multidão reunida em frente à
residência de Craxi arremessou moedas e pedras quando ele deixou sua
casa para atender uma entrevista na televisão. Em julho de 1994, novo
decreto legislativo, exarado pelo governo do primeiro-ministro Silvio
Berlusconi, aboliu a prisão pré-julgamento para categorias específicas
de crimes, inclusive para corrupção ativa e passiva. A equipe de
procuradores da operação \textit{mani pulite} ameaçou renunciar coletivamente a
seus cargos. Novamente, a reação popular, com vigílias perante as Cortes
judiciais milanesas, foi essencial para a rejeição da medida.

Na verdade, é ingenuidade pensar que processos criminais eficazes contra
figuras poderosas, como autoridades governamentais ou empresários,
possam ser conduzidos normalmente, sem reações. Um Judiciário
independente, tanto de pressões externas como internas, é condição
necessária para suportar ações judiciais da espécie. Entretanto, a
opinião pública, como ilustra o exemplo italiano, é também essencial
para o êxito da ação judicial.''\footnote{``Considerações sobre a
  operação \textit{mani pulite}'', Revista do Conselho de Justiça Federal, no.
  26, jul/set. 2004.}
\end{quote}

O fato público novo é que o juiz avança um significante político,
importante, \emph{até então inexistente no Brasil}. Ideologia de
equilíbrio do sistema, prática moderadora da indecência do capitalismo
sem riscos entre nós, ou nova realidade das coisas do poder no Brasil, o
fato é que \emph{todos os diretores e executivos mais importantes das
empreiteiras mais ricas do país foram presos por ele, e mantidos na
cadeia durante meses, até serem julgados, e, na medida das provas,
condenados}. O novo termo, meio espantoso, da política e economia
brasileira é \emph{milionários e poderosos vão para a cadeia no Brasil}.

Com base no trabalho de Polícia Federal e de Ministério Público, nos
desdobramentos de sua prática jurídica renovada, Moro conseguiu aquilo
que a esquerda brasileira sempre sonhara, mas que havia deixado de
desejar: prendeu um militar brasileiro de alta patente, o vice-almirante
Othon Luiz Pinheiro da Silva, denunciado como tendo recebido propina de
4,5 milhões de reais quando da sua administração na Eletrobras
Termonuclear -- os promotores públicos da Lava Jato avaliam que o
almirante deva ter recebido 30 milhões de reais… e, no mais castiço
estilo brasileiro, o militar recebeu a Polícia Federal com ameaças de
simplesmente ``meter bala''.\footnote{http:goo.gl/VhmJYh}

Olhando de um certo ponto de vista, de longa duração, este processo
verdadeiramente inédito no país -- que prende igualmente de
arquimilionários empreiteiros à militares de alta hierarquia -- é
possível dizer , à luz deste trabalho político da justiça que entrou em
cena na configuração de uma democracia moderna, e sua dinâmica de
compensações internas, que \emph{finalmente a ditadura militar de 1964
acabou no Brasil}. Esta é, sem dúvida, uma perspectiva limitada, mas nos
possibilita intuir as possibilidades verdadeiramente impensadas que as
novas ações da justiça podem originar.

O resultado do processo de justiça, de Moro, que está em andamento, é
forte: Dalton dos Santos Avancini, ex-Presidente da Camargo Corrêa, e
Eduardo Leite, ex-vice-Presidente da empresa, pegaram 15 anos e dez
meses de reclusão, mais multas milionárias; após dois anos de cadeia,
eles devem se beneficiar de um regime domiciliar, pelo acordo de delação
que realizaram…, enquanto João Ricardo Auler, ex-Presidente do
Conselho de Administração da empreiteira, pegou nove anos e seis meses
de cadeia, sem direito a benefícios, por não ter aderido à delação
premiada. Léo Pinheiro, ex-Presidente da empreiteira \versal{OAS}, e Agenor
Medeiros, ex-diretor, foram condenados a 16 anos e 4 meses de prisão.
Sem benefícios.

O que se observa, no manejo da justiça e da lei, é que a diferença de
destino penal é enorme -- essencialmente a possibilidade de não cumprir a
pena na cadeia -- entre quem realiza um acordo de delação e quem não o
faz. O uso da norma do acordo, deste modo, tende a abrir mão das penas
-- também praticamente inexistentes quando imperava a lei do silencio
entre os envolvidos -- para valorizar o processo judicial como
desbaratamento \emph{do próprio esquema} privado-público-político de
corrupção. De certo modo \emph{é o todo} e o público que é visado, mais
do que a justiça pessoal e de vingança \emph{contra o homem}, como, de
resto, costuma funcionar a justiça, de modo draconiano, contra os pobres
no Brasil.

Paradoxalmente, esta ação afirmativa da justiça de Sergio Moro só foi
possível porque a Presidente Dilma Rousseff regulamentou, em agosto de
2013 -- dois meses após os levantes de maio… -- obrigada por motivos
de imagem política a mostrar serviço contra as reiteradas denúncias de
corrupção em seu governo, o instituto, existente, mas até então inútil,
da \emph{delação premiada}. É uma clara dialética interna da democracia,
muito típica das mazelas petistas em seus anos de governo: acossado por
denúncias, visando à opinião pública, o governo cria o instrumento que
vai liquidá-lo, judicialmente, junto à própria opinião pública.

Fazendo casar prisão preventiva de homens poderosos -- econômica e
politicamente -- que poderiam intervir nos processos, com a novidade
jurídica da delação premiada, tendo como base e horizonte as condenações
e prisões de políticos, banqueiros, intermediários e publicitários
realizadas no processo do Mensalão, Moro realizou um verdadeiro
\emph{strike} no sistema de corrupção privado-publico-político da
Petrobras.

Dando voz a um republicanismo isento e rigoroso, o juiz legitima a sua
verdadeira ação de poder político, e de efeito social, com o termo puro
da lei, contra o desequilíbrio próprio da política: ``A corrupção não
tem cores partidárias. Não é monopólio de agremiações políticas ou de
governos específicos. Combatê-la deve ser bandeira da esquerda e da
direita.''\footnote{``Não é dos astros a culpa'', Folha de S. Paulo,
  24/8/2014.}

Todavia, como dito em seu texto sobre a \emph{mani pulite} italiana, o
juiz Sergio Moro não trabalha apenas com a percepção do aspecto técnico
da sua ação. Ele tem uma ideia muito precisa do processo mais amplo de
\emph{deslegitimação} do sistema político, e da dimensão forte de
pressões e contra pressões públicas e políticas, que ela de fato
implica.

  \section{Tradição, corrupção e
  \versal{\versal{PT}}}\label{tradiuxe7uxe3o-corrupuxe7uxe3o-e-pt}

Ao contrário do que imaginam os anticomunistas do nada, vendendo como
pão de hoje as boas ideias de 1959, os erros do governo petista se deram
inteiramente dentro do quadro de distorções e iniquidades bem próprios
do capitalismo brasileiro, que também lhes pertence de pleno direito. O
governo foi gradualmente corroído pelos efeitos graves da sua adesão ao
legítimo modo brasileiro, tradicional, e geral, de fazer política. No
Brasil, capitalismo e democracia formal -- do presidencialismo de
coalisão com os 32 partidos existentes, com as campanhas políticas mais
caras do mundo -- podem perfeitamente repor as estruturas arcaicas
nacionais, ainda muito bem incrustradas no pacto capital-estado de
caráter antissocial, de tipo antigo. A ocupação generalizada do Estado
por \emph{máfias de governo}, que atualizaram de modo pós-moderno o
tradicional Estado patrimonialista brasileiro, ligadas elas próprias aos
grandes interesses e grupos econômicos, parece ser uma constante
nacional -- certamente presente ao menos desde a ditadura militar -- e o
jogo pesado petista para o poder apenas confirmou este sistema de
razões, de longa duração.

A ideia do governo deste novo tipo de esquerda era a de que o poder
ideológico do populismo de mercado de Lula contrabalançaria, como ganho
popular e de relativa paz social, como felicidade coletiva, gerida
também pelo corpo carismático do político, a máquina infernal dos ganhos
\emph{sem contabilização} das múltiplas e várias máfias de governo, e de
Estado, brasileiras, clube privado de controle do que é público que o
partido confirmava e também passava a participar.

Tencionando com a democracia institucional, a partir do avanço histórico
constante de Ministério Público, Polícia Federal e, principalmente, de
uma Justiça que pela primeira vez atingiu imensos interesses criminosos
do pacto brasileiro capital-Estado, o partido de esquerda recém chegado
ao poder se viu, surpreendentemente para a consciência que fazia de si
próprio, responsabilizado pela história crônica da corrupção brasileira.

Ao longo de seu longo tempo de governo algo deu errado para o \versal{PT}. Na
tensão dialética interna da democracia, o seu próprio discurso
maniqueísta a respeito de ser perseguido pelas elites entrou em crise
radical, que ninguém no partido consegue pensar, deslocar ou
transformar. Bem ao contrário do que diz, o \versal{PT} está sendo processado
politicamente exatamente pela revelação do seu pacto com as grandes
elites empresariais brasileiras, as grandes empreiteiras eleitas de
sempre, prática que apenas confirmava o controle privado da gestão
pública nacional.

Fazendo precisamente o que todos sempre fizeram, o \versal{PT} precisou
responder, em juízo, aos anos que pregou intensamente contra o próprio
sistema a que aderiu. A verdadeira diferença, no jogo tenso das forças
de uma democracia em funcionamento, se deu no impensável
\emph{funcionamento crítico} da justiça brasileira -- o trabalho técnico
avançado do quase revolucionário, para o Brasil, juiz Sérgio Moro -- uma
vez que, até o processo do mensalão e a prisão preventiva por quase seis
meses de vinte diretores e donos das maiores empreiteiras do país -- entre
elas \versal{OAS}, Camargo Correa, Odebrecht, Mendes Jr., Queiróz Galvão, \versal{UTC},
Engevix, Iesa, Galvão Engenharia, Andrade e Gutierrez
…-- a nossa justiça sempre soube garantir a impunidade de abastados no
país. Impunidade que, vergonhosamente para a sua história, o \versal{PT} demandou
para si próprio, baseado na tradição brasileira, quando as coisas
começaram a dar erradas para o partido. E esta é uma verdade profunda,
do sentido oculto na famosa fotografia, do cumprimento cordial entre
Lula e Paulo Maluf.\footnote{``Muitos ficaram revoltados com a
  fotografia de Lula cumprimentando Maluf no jardim de sua mansão.
  Desfaçatez com a história. \emph{Insolência conservadora}. A
  desautorização simbólica bem radical que o homem do poder deseja
  realizar sobre os seus. Alguma tardia resistência de valor interno, de
  alguma natureza de vínculo com a história, se elevou dentro das
  pessoas. Mas o desprezo contido no gesto, da ex-esquerda com a
  ex-direita comemorando a sua nova igualdade, já demonstra que tais
  pruridos são de fato anacrônicos. O poder demanda adesão total, e a
  conversão da história na imagem que aceita tudo é sua arma principal.
  Estes muchochos de pessoas ao redor não interessam minimamente ao
  estado do poder, inclusive porque eles vêm de pessoas que julgam a
  política da imagem, mas não o conceito da política.''; escrevi na
  época do famigerado episódio. Em \emph{ensaio, fragmento}, São Paulo:
  Editora 34, 2014, p. 34.}

Afastado dos seus próprios princípios e bases críticas de classe média,
para muitos o \versal{PT} contribui para o avanço da democracia com o seu próprio
esfacelamento, entregando aos inimigos políticos a posição de virtudes
públicas que eles nunca tiveram, e que de fato não têm. Paradoxalmente,
pela própria fraqueza, o \versal{PT} daria origem a um novo patamar de
consciência pública do sentido da corrupção brasileira e do que é
democracia por aqui. Resta saber se a sanha punitiva legítima
antipetista alcançará também um dia partidos burgueses mais
tradicionais, como \versal{PSDB} e \versal{PMDB}… e os mesmos esquemas de carteis
milionários -- como, por exemplo, o de vinte anos do metrô de São Paulo
-- associados a estes partidos. No jogo de sete erros dos \emph{Esaús} e
dos \emph{Jacós} pósmodernos locais, \emph{os nossos corruptos são
sempre melhores que os dos outros}.

De fato, pelo destino vergonhoso do processo do mensalão tucano mineiro,
tudo indica que o \versal{PT} deve servir realmente de bode expiatório --
ideológico e mágico -- dos processos perigosos de uma democracia liberal
que, salvo engano, entrou em funcionamento.

  \section{Entropia, anomia e nova
  ordem}\label{entropia-anomia-e-nova-ordem}

Este muito complexo jogo de forças, todas politizadas ao extremo, todas
falando em nome de um interesse que se tornou absolutamente particular,
encarnado e único, gradualmente produziu uma verdadeira dissipação do
\emph{lugar do governo}. Se deu no Brasil o vazio de alguma
\emph{integração por um governo}. A estrutura de uma hierarquia
simbólica que oriente as ações gerais e coletivas por um ponto de fuga
no poder Executivo, e seu chefe, desapareceu no Brasil ao longo dos
primeiros meses do ano de 2015. Descobrimos que, ao menos por aqui, o
governo é um pacto simbólico construído em um espaço político que
\emph{não coincide com a legitimidade legal institucional}. E este pacto
foi verdadeiramente rompido, fragmentado, dissolvido, fazendo desaparecer
\emph{o lugar do governo} no sistema geral da política e do poder, que
deixou de fato de existir por um segundo histórico.

E faz parte da falência do governo a estranha falência do Partido do
governo, que só pode ser creditada ao efeito da política insólita do
lulismo sobre ele, exatamente quando aquele Partido conquistou a quarta
vitória nacional consecutiva… \emph{``Ao vencedor…''}

No horizonte mais longínquo deste processo está a história de
instabilidade institucional e luta radical pelo poder no país
pós-colonial e economicamente periférico. Uma história de longa duração,
de degladiação das elites pelo espólio de um lugar no mundo mal
constituído: crises institucionais fortes presentes na origem da nação,
crises ocorridas na sua integração territorial, crises no processo de
transformação da economia de base escravista, crises profundas no
processo já tardio de expansão da indústria, crises, atuais, na
renovação do sistema do capitalismo mundial rumo a sua financeirização
total. Cada novo lance histórico de ajuste local a alguma nova ordem de
funcionamento das circulações mundiais de riquezas que passam pelo
Brasil sempre teve efeito político que poderia alcançar \emph{a própria
estrutura da base institucional fraca do país}, a estrutura do pacto
muito instável e superficial das elites brasileiras pelo seu próprio
poder. Este processo de grande escala, da situação histórica do país
atrasado pós-colonial em atualização modernizante constante, produz uma
série de momentos políticos \emph{negativos,} síncopes históricas, que
chamei, em outro lugar, de \emph{o conceito do transe}, brasileiro:

\begin{quote}
``A ideia brasileira do \emph{transe}: a estrutura insólita da política
no país antiga colônia, escravista e atrasado. O andamento acelerado da
modernização deste espaço social por vezes se dá aos saltos do
\emph{transe} político e social. Na hora decisiva o país não sabe se vai
para a frente ou para trás, porque, de fato, pode ir tanto para a frente
quanto para trás. O \emph{transe} é também a própria reversão
conservadora do processo da mudança. A crise ``a todo transe'', já dizia
Nabuco, em uma página que, decorada por Glauber Rocha, se tornou o
sonho, o pesadelo, do giro infinito da história que não passa, mas
também não começa: \emph{Terra em transe}.''\footnote{\emph{ensaio,
  fragmento,} São Paulo: Editora 34, 2014, p.49.}
\end{quote}


A situação do presente é um fenômeno político de grande porte. A perda
da integridade do poder erigido, em meio do próprio processo que o
constituiu. Uma aceleração histórica em que os agentes de todas as
dimensões, que se multiplicam, e que, ao se moverem, simplesmente
\emph{corroem} o ponto de fuga do poder do governo, alcançou o estado de
uma dinâmica entrópica, dissipativa, tendente, do ponto de vista do
governo, a igualar o poder da Presidente ao de qualquer outro presente
no sistema, em ebulição política.

A multiplicação de forças ao máximo, a dissipação da legitimidade de um
foco ordenador para o sistema geral, em fragmentação, produziu uma vida
política sem centro, e sem a fantasia política organizadora de algum
projeto. Talvez a dissipação da ``integridade egoica'' do governo, a
fantasia de uma organização, de um horizonte de desejo, de uma forma
qualquer a alcançar, seja o ponto final para a hiper fragmentação, das
forças e das vozes, a nova modalidade de \emph{transe local} -- em uma
imagem das coisas políticas correspondente, em um outro estágio, à
análise do pré-1964 do famoso filme. No Brasil do início do quarto
governo petista, desapareceu, através da luta política extremada,
\emph{a fantasia de integridade} de um governo, levando a zero o seu
poder frente às demais forças presentes.

Porém, muito diferente de 1964, não há no próprio horizonte do processo
da multiplicação e redução do poder às vozes e corpos, nenhuma força que
possa se arvorar a novo núcleo de autoridade, exército ou projeto
neo-imperialista central, como existia de fato, dentro e fora do país,
nos anos de 1960. Não há, também no campo da oposição ao governo, nenhum
inimigo \emph{que de fato deseje alguma outra coisa}, como existia na
conspiração e na busca ativa de poder e ditadura, do pacto
militar-civil, pró-americano, de 1964. Este conjunto de forças, antigas,
está fora do atual cenário de \emph{transe} das vozes, anomia política e
intensidade entrópica das forças de hoje, embora, para dar o sabor de
absurdo muito próprio ao processo brasileiro, elas sejam também evocadas
e pedidas nas ruas por alguma voz, em uma espécie de reserva autoritária
limite das coisas brasileiras.

A somatória acumulativa dos seguintes processos, cada um deles em si
mesmos complexos, levou praticamente a zero o grau de uso do poder
político do governo de Dilma Rousseff: 1. profundos arcaísmos
brasileiros, 2. ideologia antipetista, 3. interesses políticos
eleitorais e concretos, 4. crise econômica real, com base na recolha do
capital mundial e no estouro da \emph{bolha Brasil} lulista 5. sistema
político degradado (32 partidos etc…), 6. efetiva crise moral petista
com importantes condenações na justiça, 7. dissipação de um grande
esquema privado-público-político de corrupção, com dissolução do lugar
de poder de todo um setor da burguesia nacional, as grandes construtoras
-- aliado ao governo petista -- 8. crise interna continuada do próprio
\versal{PT} frente ao seu governo, 9. fraqueza do caráter político da presidente,
com ausência de qualquer carisma e falta de ligação orgânica com
qualquer setor social, 10. emergência, neste quadro, de uma liderança à
direita no Congresso, oportunista, agressiva e \emph{produtiva}, 11.
nova judicialização da política, ou criminalização das práticas
tradicionais de governo e burguesia, em escala nunca antes imaginada,
12. nova conquista das ruas por movimentos populares de direita,
reorganizados no país por um sistema de comunicações original,
constituído na internet e 13. ausência real no espaço público de
movimentos organizados e expressões à esquerda do espectro político,
fragmentados e desmobilizados por três processos: a adesão e agregação
burocratizantes dos grande movimentos sociais (\versal{CUT}, \versal{MST} etc…)
cooptados por lugares na gestão dos governos petistas; o afastamento
simbólico de uma esquerda crítica, mínima, ao modo de ser do governo
petista; e a fragmentação -- no limite de \emph{um anarquismo de estilo
de vida} -- de muito pequena escala, dos movimentos jovens de militância
à esquerda, de busca de ação direta, simultaneamente crítica ao governo
dito de esquerda.

Este amplo quadro de múltiplas crises acumuladas, em frentes demais para
a ação de um governo ele mesmo em crise de \emph{mudança de matriz
econômica}, levou ao esvaziamento do poder político da presidente
durante todos os primeiros meses de seu segundo mandato. Ela cedeu o
poder de decisão econômica à sua frente neoliberal interna, dando
poderes de decisão amplos ao seu ministro da fazenda Joaquim Levy -- mas
ao mesmo tempo, estranhamente, desautorizando-o… -- homem advindo do
grande mercado financeiro brasileiro ligado ao Bradesco, o maior banco
privado do país.\footnote{De fato as ações econômicas do governo no
  período foram em geral erráticas e desorganizadas, revelando a crise
  interna entre a política fiscalista de Levy, e o desejo petista,
  impotente, de manutenção de direitos, e de gasto social, tornado
  inviável no curto prazo. Mais uma vez o governo se viu dividido.} E
cedeu a negociação política inteiramente ao vicepresidente peemedebista
Michel Temer, representante de elites conservadoras que tradicionalmente
negociam com a política preferencialmente para o próprio enriquecimento.
Este movimento de cessão de poder, em que a presidente demonstrava a
fraqueza de seu próprio lugar de direção no processo, acabaria por
multiplicar a fragmentação das forças, elevando ao quadrado a própria
crise em um jogo tendente à crise, do mundo do \emph{todos contra todos}
pós-moderno brasileiro, que a fragmentação dos interesses acabou por
produzir.

Dilma terminou no mesmo nível de poder de cada agente que se
manifestasse no seu processo, e todos podiam dispor de seu governo, cada
um decidindo por si mesmo a efetividade de um impeachment: as ruas
querendo gritar a sua solução moralista seletiva à direita, derrubando-a
– o trabalho de ativismo político, algo profissional, daquela organização
pública à direita tornou \emph{verossímil}, como dizia Vico 
sobre alguns lances da história, a solução traumática – Eduardo Cunha animando as ruas que animam Cunha a
derrubá-la no Congresso, o Superior Tribunal Eleitoral podendo decidir,
no panorama dado, a invalidação da sua eleição, o Tribunal de Contas da
União podendo decidir pela recusa das suas contas de governo, dando
prova de crime de responsabilidade para a abertura do processo de
impeachment, a Câmara aprovando -- com a lamentável e degradante ação
oportunista de \versal{PSDB} e \versal{PMDB} -- matérias de interesses muito particulares
que aumentam o gasto público e ao mesmo tempo degradam ainda mais a
legitimidade do governo, o juiz Sérgio Moro podendo detonar
simbolicamente o processo de impeachment, dependendo de suas novas
decisões a respeito da corrupção petista na Petrobras…

Enfim, cada força parece ter se autonomizado, no máximo de sua potência,
reduzindo ao máximo, no mesmo movimento, a força da \emph{fantasia
integradora} do poder executivo e de sua chefe de governo. Assim se
produziu a \emph{anomia política} tendente à entropia brasileira neste
momento. Tal movimento amplo produziu a situação política extrema de
\emph{um governo recém-eleito que não pode governar}.

E é este processo, em um nível que ninguém pode pensar, que se liga à
``entropia do capitalismo, em seu núcleo orgânico, que desorganiza até
mesmo as suas forças antissistêmicas'', lembrado por Paulo Arantes.

Em outro panorama histórico, e isto é uma fantasmagoria do passado
brasileiro que assombra certo pensamento à esquerda, estariam de fato
plenamente criadas as condições para a derrubada do governo por um
golpe, por exemplo, militar. Neste momento, no entanto, não existe
interesses de horizonte estratégico para o desenvolvimento capitalista
no Brasil nesta direção. No último lance no tabuleiro da anomia política
brasileira que vou anotar aqui, nas primeiras semanas do mês de agosto
de 2015, foi possível observarmos um apelo \emph{por cima} de retorno à
ordem, diante do grau zero de poder, do corpo político à deriva da
Presidente, e do risco da derrocada final de seu governo pela convocação
organizada de uma nova manifestação de massas nacional, agora pedindo
\emph{apenas seu impeachment} -- bem como a prisão de Lula…, a voz do
homem de massa qualquer, organizado, se tornou por fim poder decisivo,
como todos os demais núcleos de poder do país, em uma dissolvência
estratégica do poder nas ruas, à direita, do país --.

A convocação por cima uniu alguns agentes sociais especiais, os mais
puros e fortes poderes estruturados do capitalismo à brasileira, neste
momento para salvar a Presidente e o seu governo. A Folha de S. Paulo
escreveu editoriais pedindo o fim da crise política, visando a evitar o
aumento da degradação econômica do Brasil; o presidente do Bradesco,
Luiz Carlos Trabuco, veio a público, em entrevista no mesmo jornal, e
pediu ``grandeza'' dos agentes políticos, que deixassem os próprios
interesses fragmentários, em sinalização entendida à favor do governo;
líderes e organizações de classe empresariais -- Fiesp, \versal{CNI}… --
produziram comunicações, mesmo que a contra gosto, pelo fim da crise
política, e por fim, e talvez principalmente, coroando a onda:

\begin{quote}
``O vice-presidente do Grupo Globo, João Roberto Marinho, procurou nas
últimas semanas líderes das principais forças políticas do país e
integrantes do governo para expressar preocupação com a crise e pedir
moderação para evitar que ela se aprofunde ainda mais. (…) Conforme
relatos obtidos pela Folha sobre estas conversas, Marinho
manifestou em todos os encontros preocupação com a situação econômica,
mencionando a queda acentuada do faturamento dos grupos de mídia e de
outros setores da economia. (…) Outros líderes empresariais
transmitiram mensagens semelhantes nas últimas semanas, mas os apelos de
Marinho tiveram ressonância maior entre os políticos por causa da
influência da Globo na opinião pública. (…) Nos últimos dias, porém, o
governo começou a discutir com líderes do \versal{PMDB} no Senado uma agenda de
reformas e ganhou fôlego para enfrentar os opositores que defendem a
saída de Dilma como solução para a crise.''\footnote{``Grupo Globo pediu
  moderação a políticos'', por Daniela Lima, em Folha de S. Paulo, 16 de
  agosto de 2015.}
\end{quote}

Assim, tudo indica, se as coisas forem enfim nesta direção, o governo
petista foi derrubado e foi reposto ao longo do primeiro semestre de
2015. Derrubado pela somatória de forças anômicas bastante agressivas
acumuladas à direita, incluindo entre elas os próprios grandes erros. E
resposto na sua legitimidade, na hora da maior entropia, pelo cacife
final das maiores forças econômicas estruturadas, representadas pelos
grandes capitalistas brasileiros.

O lastro final da governabilidade do quarto mandato de governos
petistas, se acontecer, será dado por homens como João Roberto Marinho e
Luis Carlos Trabuco,\footnote{Vejamos as posturas semelhantes de Roberto
  Setubal, presidente do Itaú Unibanco, já no adiantado da hora da
  crise: ``Nada do que vi ou ouvi até agora me faz pensar que há
  condições para um impeachment. Por corrupção, até aqui, não tem
  cabimento. (…) Pelo contrário, o que a gente vê é que Dilma permitiu
  uma investigação total sobre o tema. Era difícil imaginar no Brasil
  uma investigação com tanta independência. (…) Seria um
  artificialismo tirar a presidente neste momento. Criaria uma
  instabilidade ruim para nossa democracia. (…) Não se pode tirar um
  presidente do cargo porque momentaneamente ele está impopular. É
  preciso respeitar as regras do jogo, precisa respeitar a Constituição.
  Eu sou a favor da constituição.'' Entrevista à David Friedlander,
  Folha de S. Paulo, 23 de agosto de 2015. De fato o processo todo, da
  crise com os bancos de 2012, à reposição do governo pelos próprios
  grandes banqueiros e empresários, a partir de julho/agosto de 2015,
  parece ter sido uma \emph{viagem redonda} cujo sentido político final
  foi a destruição do projeto petista de governo, e, por um tempo
  indeterminado a partir de agora, a destruição do próprio \versal{PT} e seu
  lugar na política brasileira. {[}\emph{nota acrescentada ao livro no
  prelo}{]}} que no auge da crise de esvaziamento do \emph{lugar do
poder} do governo decidiram por sua continuidade, certamente avaliando
os próprios interesses. Deste modo, deve-se a estas maiores forças do
Capital nacional -- com seus desdobramentos de interesses globalizados
-- a reposição simbólica da estrutura institucional da própria
democracia, e o lastro de legitimidade de manutenção do último governo
petista. O governo seria salvo, assim, \emph{pelo Capital em pessoa}.

\emph{Estamos em pleno mar,} e se o outro significante que busca
totalizar o processo, o da ilegitimidade absoluta do governo por causa
da crise econômica e da corrupção revelada da operação Lava Jato -- que
as massas organizadas de direita e de classe média gritam a plenos
pulmões na rua, com grande \emph{pathos} de intolerância -- não chegar a
se impor ao sistema político em ebulição, será esta a decisão das forças
sociais maiores e mais ricas do país.

E se assim for, teremos que admitir que nenhum movimento social, nenhuma
organização de trabalhadores, nenhuma crítica à esquerda, nenhuma
mobilização popular pela legitimidade democrática do governo, mantiveram
o seu lugar de poder constituído; mas sim, no momento de grau zero de
seu poder, o grande Capital nacional resolveu, até segunda ordem,
mantê-lo como agente e dispositivo legítimo no tenso jogo político
social brasileiro.

Globo, Bradesco e Itaú, entre outras forças empresariais estruturadas,
que no passado foram forças maiores de investimento e interesse em uma
ditadura no país, hoje, no auge da crise que também alimentaram,
preferem, quase sozinhos socialmente, como \emph{sujeitos preferenciais}
no processo \emph{de todos contra todos}, a manutenção democrática do
governo petista terminal. Eles tentam repor o pacto de uma fantasia
unificadora possível de governo. São forças integradoras da democracia
brasileira em crise, e o governo petista, se conseguir se livrar de
todos os múltiplos inimigos cultivados no limite das suas forças, deverá
a estes imensos poderes brasileiros a manutenção da própria cabeça.

Por fim, com o esgarçamento de todo o sistema político brasileiro,
podemos observar de modo nítido, quase cru e nu, quais são os
verdadeiros sujeitos das decisões finais, em uma democracia capitalista
de massas periférica, apenas funcionando o caos de sua própria produção.

\bigskip\hfill\begin{minipage}{.7\textwidth}
\emph{São Paulo, entre a primeira manifestação nacional pelo impeachment
de Dilma Rousseff, em março, e a terceira, em agosto, de 2015.}
\end{minipage}
