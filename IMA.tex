\addcontentsline{toc}{chapter}{Fotos de João Bittar}

\part*{Fotos de João Bittar}

\includepdf[noautoscale,pages=1-11]{ima.pdf}

%Para pagina em branco ao lado da foto do \versal{\versal{\versal{\versal{LUL}}}}a apontando o umbigo.

%“Com certeza a foto mais importante que eu fiz na minha vida”

%Identificação das fotos numeradas começando pela do umbigo.
\section{Legendas das fotos}

\thispagestyle{empty}

\begin{enumerate}\parsep=0em\itemsep=0em\leftskip=-1em\normalsize
\item Lula no \textsc{x} Congresso dos Metalúrgicos. Poços de Caldas\\ (25/04/1979)

\item Greve dos metalúrgicos, porta de fábrica. (15/03/1978)

\item Intervenção policial no Sindicato dos Metalúrgicos.\\ (15/03/1978)

\item Estádio de Vila Euclides (atual Estádio Municipal 1º de Maio) em
São Bernardo do Campo, palco das assembleias dos movimentos sindicais do \textsc{abc}
paulista. Um dos maiores atos públicos com a participação de cerca de 150 mil
pessoas. A intervenção no sindicato foi suspensa no dia 15 de maio e a
diretoria do sindicato reassume oficialmente no dia 18 de maio de 1979.  (14/05/1979) 

\item Em lados opostos da mesa de negociação salarial, Luiz Inácio Lula da Silva e
Luís Eulálio de Bueno Vidigal. (15/03/1978)

\item Lula é carregado pelos metalúrgicos em greve. (02/04/1980)

\item Lula reassume o sindicato apos intervenção. (18/01/1979)

\item Sindicato convoca a categoria para assembleia no Estádio de Vila Euclides. (13/05/1979)

\item Congresso dos Metalúrgicos de \textsc{sbc}, realizado no Guarujá.  “Hoje eu não tô
bom”, dizia João Ferrador, na camiseta de Lula. Este personagem foi criado pelo
jornal a \textit{A Tribuna Metalúrgica} para atrair a atenção, passar a
mensagem e fugir da censura militar. (08/10/1978)

\item Lula prepara sua receita de coelho no fogão a lenha do seu rancho “Los Fubangos”, à beira da
represa Billings, em \textsc{sbc}. ``Fubango'' é gíria para se
referir a uma pessoa mal arrumada e/ou sem dinheiro. (07/04/2001)

\item Lula com Marisa no sítio “Los Fubangos”.  (07/04/2001)

\item Lula tirando leite de cabra em seu sítio. (07/04/2001)
\end{enumerate}

\includepdf[noautoscale,pages=13]{ima.pdf}
