\begin{itemize}

\item Dilma Rousseff e o ódio político, escrito \emph{no calor da hora}, é um
  retrato sintético, que procura retornar às origens da crise social e política
  que marcou o primeiro governo da \emph{presidenta} Dilma.  Retomando a
  reflexão anterior sobre as funções da personalidade que apresentei em
  \emph{Lulismo, carisma pop e cultura anticrítica} (2011), elenquei neste ensaio
  diversos fatores que contribuíram para a radicazação da crise do governo
  desde o início do segundo mandato, na tentativa de compreender, por um lado,
  a falência do Partido dos Trabalhadores — indiscriminado do ocaso econômico
  do projeto lulista — e, por outro, a emergência de uma nova direita
  organizada no país, cujo ideário, em uma das suas facetas, foi capaz de
  recuperar a voz mais radical do conservadorismo brasileiro, antipopular,
  invocando inclusive um comunismo inexistente como base de estratégia de
  direito ao ódio como política.  

\item Tales Ab’Sáber, psicanalista e ensaísta, é professor de Filosofia da Psicanálise da Universidade Federal 
de São Paulo (\textsc{unifesp}), autor de 
\emph{Lulismo, carisma pop e cultura anticrítica},
\textit{O~sonhar
restaurado: formas do sonhar em Bion, Winnicott e Freud} e
\textit{A música do tempo infinito}. 

\end{itemize}

