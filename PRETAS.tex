\begin{itemize}


\item \textbf{Dilma Rousseff e o ódio político}, escrito \emph{no calor da
hora}, é um retrato sintético, que procura retornar às origens da crise social e
política que marcou o primeiro governo da \emph{presidenta} Dilma. O texto
compreende, por um lado, a falência do Partido dos Trabalhadores em produzir uma
política que lhe fosse minimamente favorável, na medida em que o partido
tornou-se indiscriminado do ocaso econômico do projeto lulista. E, por outro, a
emergência de uma nova direita organizada no país, e seu ideário que, em uma das
suas facetas, recupera a voz do conservadorismo brasileiro mais radical, que
fetichiza um comunismo inexistente como base de uma estratégia de direito ao
ódio como política. 
  
\item \textbf{Tales Ab’Sáber}, psicanalista e ensaísta, é professor de Filosofia da Psicanálise da Universidade Federal 
de São Paulo (\textsc{unifesp}), autor de 
\emph{Lulismo, carisma pop e cultura anticrítica},
\textit{O~sonhar
restaurado: formas do sonhar em Bion, Winnicott e Freud} e
\textit{A música do tempo infinito}. 

\end{itemize}

