\begin{resumopage}

\item[Lulismo, carisma pop e cultura anticrítica]
faz um balanço das condições políticas e da natureza do pacto social realizado no
governo Lula, que impulsionou a renovação do carisma do líder petista. É também
uma avaliação da realização final de seu poder carismático através da expansão
global da sua forma em uma nova ordem simbólica, realizada por meio de uma
convergência de interesses nos fóruns de indústria cultural e econômicos
internacionais, que visavam posicionar o Brasil contemporâneo. Tal movimento
histórico é pensado aqui como a emergência de um \textit{carisma pop}, uma ordem
avançada de dominação política, em que a figura do homem público é investida
dos poderes próprios da forma \textit{mercadoria} e seu fetichismo endógeno.

\item[Tales Ab’Sáber,] psicanalista e ensaísta, é membro do Departamento de Psicanálise 
do Instituto Sedes Sapienti\ae{}, professor de Filosofia da Psicanálise da Universidade Federal 
de São Paulo (\textsc{unifesp}), autor de \textit{O~sonhar
restaurado: formas do sonhar em Bion, Winnicott e Freud} (Editora~34, 2005) e
\textit{A música do tempo infinito} (CosacNaify), no prelo. 

\item[João Bittar] (1951--2011) foi criador da importante agência de fotojornalismo Angular 
na década de 1980. Ao longo de mais de 40 anos dedicados à fotografia, Bittar foi fotógrafo e 
editor das maiores revistas e jornais do país, como \textit{Veja}, \textit{Isto é}, \textit{Gazeta Mercantil}, 
e  \textit{Época}, além de ter sido o responsável pela implantação do processo de digitalização fotográfica 
na \textit{Folha de São Paulo}. Bittar formou várias gerações de fotógrafos, tais como Marlene Bergamo, 
Chico Ferreira, Egberto Nogueira, Rogerio Assis e Apú Gomes.
\end{resumopage}

